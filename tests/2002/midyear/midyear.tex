\documentclass[a4paper,10pt]{article}
\setlength{\parindent}{0ex}
\setlength{\parskip}{1.5ex}
\usepackage{Defn201}
\usepackage[dvips]{graphicx}
\usepackage{epsfig}
\usepackage[a4paper, hmargin=25mm, vmargin=30mm, nohead]{geometry}



\begin{document}
\newcommand{\marks}[1]
{\begin{flushright}{\bf (#1 marks)}\end{flushright}}

{\centering \large \bf THE UNIVERSITY OF WAIKATO\\}
{\centering \large \bf Department of Computer Science\\[0.5cm]}

{\centering \large \bf 0657.201Y Computer Systems~\CORYEAR \\}
{\centering \large \bf Mid Year Test~\MIDTESTDATE \\[1cm]}

\begin{itemize}
  \item Please answer ALL questions.

  \item Time allowed: 90 minutes

  \item Marked out of: 90

  \item Contribution toward final grade: \MIDTESTWORTH
  
  \item Please answer questions on the answer sheet provided. 

  \item Write your name and student ID on each page

  \item This is a CLOSED book test, although calculators are permitted
\end{itemize}

\hrule
\section{Multichoice}
\begin{enumerate}

\item The wave file format is used to store
\begin{enumerate}
  \item audio data
  \item vectorised graphics data
  \item bitmap graphics data
  \item character data
\end{enumerate}

\item Which of the following best describes the type of data stored in a
PostScript file ? 
\begin{enumerate}
  \item audio data
  \item vectorised graphics data
  \item bitmap graphics data
  \item character data
\end{enumerate}

\item What are the two major components of any central processing unit
(CPU) that are visible to the programmer? 
\begin{enumerate}
  \item instruction set and arithmetic logic unit 
  \item registers and instruction set
  \item arithmetic logic unit and control unit
  \item registers and control unit
\end{enumerate}

\item What is the hexadecimal number 0x123a in binary?
\begin{enumerate}
  \item 0100 0010 1100 0101
  \item 1010 0011 0001 0010
  \item 0001 0010 0011 1010
  \item 0001 0010 0011 1011
\end{enumerate}

\newpage
\item Which of the following is not a component of the central
processing unit

\begin{enumerate}
 \item arithmetic logic unit
 \item primary storage 
 \item control unit
 \item registers
\end{enumerate}

\begin{figure}[h]
\begin{footnotesize}
\begin{center}
\begin{tabular}{|lp{4cm}|lp{4cm}|}
\hline
(a) &
\begin{verbatim}
       slt   $13, $3, $4
       beqz  $13, label
       ...
label: 
       ...
\end{verbatim}
& (b) &
\begin{verbatim}
       add  $2, $0, $0
label: slei $4, $2, 3
       beqz $4, label2
       ...
       addi $2, $2, 1
       j    label
label2:
       ...     
\end{verbatim}
\\
\hline
(c) &
\begin{verbatim}
       add  $2, $0, $0
label: slti $4, $2, 3
       beqz $4, label2
       ...
       addi $2, $2, 1
       j    label
label2:
       ...
\end{verbatim}
& (d) &
\begin{verbatim}
       slt   $13, $3, $4
       beqz  $13, label
       ...
       j     label2
label:
       ...
label2:
       ...
\end{verbatim}
\\
\hline
\end{tabular}
\end{center}
\end{footnotesize}
\caption{Code for multichoice questions~\ref{ques:multicode}
--~\ref{ques:multicodeend}}
\label{fig:multicode}
\end{figure}

\item 
\label{ques:multicode}
Which of the WRAMP code segments in Figure~\ref{fig:multicode}
would implement an if statement without an else clause?
\begin{enumerate}
  \item a
  \item b
  \item c
  \item d
\end{enumerate}

\item Which of the WRAMP code segments in Figure~\ref{fig:multicode}
would implement an if statement with an else clause?
\begin{enumerate}
  \item a
  \item b
  \item c
  \item d
\end{enumerate}

\item 
\label{ques:multicodeend}
Which of the WRAMP code segments in Figure~\ref{fig:multicode} would
implement a loop which iterates three (3) times.  
\begin{enumerate}
  \item a
  \item b
  \item c
  \item d
\end{enumerate}



\item Which one of the following statements is true?
\begin{enumerate}
  \item all WRAMP machine code instructions are of equal length
  \item instructions in a memory-register architecture are generally
   shorter (require fewer bits to encode) than instructions in a
   register-register architecture 
  \item in modern CPUs' data transfers between registers within the
  CPU are significantly slower than those between registers and main
  memory. 
  \item the WRAMP processor is an example of a memory-register architecture.
\end{enumerate}

\item Assembler directives are used in an assembly language program to
\begin{enumerate}
  \item give commands to the assembler
  \item replace instructions
  \item generate instructions
  \item comment the code
\end{enumerate}

\end{enumerate}

\newpage
\section{Short Answer}

\begin{enumerate}

\item What is the uni-code coding scheme used to represent? Why was it
developed?  
\marks{4}

\item What number does 1011001 represent? Is it possible to tell? If
not, what extra information is needed? 
\marks{4}

\item What are the four major building blocks used to construct a
datapath in a CPU?
\marks{4}

\item Explain the difference between the following two WRAMP instructions:
\begin{enumerate}
    \item \texttt{lw  \$8, label(\$0)}
    \item \texttt{la  \$8, label}
\end{enumerate}
\marks{4}

\item In a computer system what is the purpose of the single
bit signal called the clock? What determines how fast the clock can be
driven? 
\marks{4}

\item Convert the following 32 bit IEEE-754 format number to decimal.
The IEEE-754 format is shown in Figure \ref{fig:ieee754}.
\begin{center}
\texttt{01000001 11011010 00000000 00000000 (0x41da0000)}
\end{center}
\marks{4}

\item Convert the decimal number 11.125 to the IEEE-754 format that is
provided in Figure \ref{fig:ieee754}.  \marks{4}

\begin{figure}[h]
\begin{center}
    \psfig{figure=float.eps, width=.5\linewidth}
    \caption{IEEE-754 Format}
    \label{fig:ieee754}
  \end{center}
\end{figure}

\item If the instruction \texttt{bnez \$3, 0x15} is located at the
address 0x100,  what memory location will the next instruction be
fetched from, if this instruction is executed and the branch is taken
(i.e. \$3 $\neq$ 0)? 
\marks{4}

\item Using the supplied WRAMP Instruction Set Architecture document,
assemble the WRAMP instruction \texttt{subui \$5, \$12, 0x123} to
machine code 
\marks{4}

\item Using the supplied WRAMP Instruction Set Architecture document,
disassemble the WRAMP machine code instruction:
\begin{center}
\texttt{1000 0101 1001 0000 0000 0000 0001 1010}
\end{center}
\marks{4}

\item The header for a GIF image is shown in Figure
\ref{fig:gifhex}. Determine in decimal the width and height of this
image. You should show any working. The file format for GIF images is
shown in Figure \ref{fig:gifheader}.  \marks{8}

\begin{figure}[h]
{\small
%This was taken from gifquest and is 96x72 pixels
\begin{verbatim}
       00000000: 4749 4638 3961 6000 4800 f700 0029 2f1c  GIF89a`.H....)/.
       00000010: 8098 783b 5e21 5360 644b 7b2b 547a 6c31  ..x;^!S`dK{+Tzl1
       00000020: 384c 4c5f 47c7 cfda 3142 2375 7d8c 929d  8LL_G...1B#u}...
       00000030: ac39 4539 5f7c 4473 8c64 5970 4460 6d7c  .9E9_|Ds.dYpD`m|
       00000040: 4271 2a53 706c 4354 3763 8b44 637c 7c42  Bq*SplCT7c.Dc||B
\end{verbatim}}
\caption{HEX dump of GIF file header}
\label{fig:gifhex}
\end{figure}

\begin{figure}[h]
  \begin{center}
  \psfig{figure=gif_defn.eps, width=.8\linewidth}
  \caption{GIF file header format}
  \label{fig:gifheader}
  \end{center}
\end{figure}

\newpage    

\item
\label{ques:swapend}

Figure~\ref{fig:swapend} contains a WRAMP program. You are to trace
the execution in order to determine what it does.  A break point has
been set at the start of the loop (line 9) to achieve this.

\begin{enumerate} 
\item What would the contents of registers 2 to 5 be each time this
break point was encountered when the program was run?

\item What does this program do?
\end{enumerate}

\begin{figure}[h]
\begin{footnotesize}
\begin{center}
\begin{tabular}{|p{10cm}|}
\hline
\begin{verbatim}
        1:   .global main
        2:   main:
        3:         lhi  $2, 0x1234
        4:         ori  $2, $2, 0x5678
        5:         addi $3, $0, 0
        6:         addi $4, $0, 0
        7:         addi $5, $0, 4
        8:   loop:    
        9:         beqz $5, endloop
        10:        slli $4, $4, 0x8
        11:        andi $3, $2, 0xff
        12:        or   $4, $4, $3
        13:        srli $2, $2, 0x8
        14:        subi $5, $5, 1
        15:        j    loop
        16:  endloop: 
        17:        jr   $ra
\end{verbatim}
\\
\hline
\end{tabular}
\end{center}
\end{footnotesize}
\caption{WRAMP program for Question~\ref{ques:swapend}}
\label{fig:swapend}
\end{figure}

\item 
\label{ques:cprog}

Figures \ref{fig:cprog} and \ref{fig:wrampgen} show the "C" code and
the WRAMP code generated by the \texttt{wcc} compiler for a non-leaf
function \verb+volume_cylinder+ that calculates the volume of a
cylinder given its diameter and height. To calculate the area of the
circle the function calls a function called \verb+area_circle+.

\begin{enumerate}
  \item For the C function indicate which lines of the WRAMP code were
  generated for:
  \begin{enumerate}
   \item \verb+radius = diameter >> 1;+
   \item \verb+area = area_circle(radius);+
   \item \verb+volume = area * height;+
  \end{enumerate}

\marks{6}

  \item Draw a diagram to show what the stack frame created when the
  function \verb+volume_cylinder+ is called will look like. On your
  diagram clearly indicate the purpose of each entry in the stack
  frame. Your diagram should also clearly indicate where the two
  parameters that are passed to the function are stored on the stack.
  \marks{8} 
\end{enumerate}

\newpage
\begin{figure}[t]
\begin{footnotesize}
\begin{center}
\begin{tabular}{|p{10cm}|}
\hline
\begin{verbatim}
        1:   int area_circle(int radius);
        2:
        3:   int volume_cylinder(int diameter, int height)
        4:   {
        5:     int radius;
        6:     int area;
        7:     int volume;
        8:     int i;
        9: 
        10:    radius = diameter >> 1;
        11:    area = area_circle(radius);
        12:    volume = area * height;
        13:    return volume;
        14:  }
\end{verbatim}
\\
\hline
\end{tabular}
\end{center}
\end{footnotesize}
\caption{C code for Question~\ref{ques:cprog}}
\label{fig:cprog}
\end{figure}

\begin{figure}[h]
\begin{footnotesize}
\begin{center}
\begin{tabular}{|p{10cm}|}
\hline
\begin{verbatim}
        1:   .global volume_cylinder
        2:   .text
        3:   volume_cylinder:
        4:         subui $sp,  $sp, 7
        5:         sw    $12,  1($sp)
        6:         sw    $13,  2($sp)
        7:         sw    $ra,  3($sp)
        8:         lw    $13,  7($sp)
        9:         srai  $13,  $13, 1
        10:        sw    $13,  6($sp)
        11:        lw    $13,  6($sp)
        12:        sw    $13,  0($sp)
        13:        jal   area_circle
        14:        addu  $13,  $0, $1
        15:        sw    $13,  5($sp)
        16:        lw    $13,  5($sp)
        17:        lw    $12,  8($sp)
        18:        mult   $13,  $13, $12
        19:        sw    $13,  4($sp)
        20:        lw    $1,   4($sp)
        21:        lw    $12,  1($sp)
        22:        lw    $13,  2($sp)
        23:        lw    $ra,  3($sp)
        24:        addui $sp,  $sp, 7
        25:        jr    $ra
\end{verbatim}
%$
\\
\hline
\end{tabular}
\end{center}
\end{footnotesize}
\caption{WRAMP code generated by \texttt{wcc} for C program in Figure
\ref{fig:cprog}}
\label{fig:wrampgen}
\end{figure}

\newpage

\item 
\label{ques:datapath}

Figure \ref{fig:wrampblok} shows the architecture of the WRAMP
CPU that contains three 32 bit internal buses. Although not shown on
the diagram there are control lines between the control unit and each
of the components, which are used to control the flow of data on the
data-path. The control signals for each of the components and their
functionality is defined in Tables \ref{table:signals} and
\ref{table:alu}.

\begin{enumerate}
 \item Explain how a value can be copied from register (e.g. \$10) in
   the register file to another register (e.g. \$11) in the register
   file on the architecture given in Figure \ref{fig:wrampblok}.
   \marks{2}

  \item If the registers contained the values shown in Figure
  \ref{fig:wrampreg} what would the new values of the registers be
  after executing the following three control steps:
{\small
\begin{verbatim}
Step 1: pc_out, alu_out, alu_func = inc, temp_in
Step 2: a_out, sel_a = 3, b_out, sel_b = 8, alu_func = sub, alu_out, sel_c = 8, c_in
Step 3: temp_out, imm_16_out,  alu_out, Func = add, pc_in
\end{verbatim}
}
	You need only fill in the values for the registers that have changed. 
\marks{6}

\end{enumerate}

\begin{figure}[h]
\begin{center}
    \psfig{figure=datapath.eps, width=.8\linewidth}
    \caption{Datapath Architecture for WRAMP CPU}
    \label{fig:wrampblok}
  \end{center}
\end{figure}

\begin{figure}[h]
\begin{center}
\begin{tabular}{|l|l|l|l|}
\hline
\verb+$0 = 0x0+ & \verb+$1 = 0x10+ & \verb+$2 = 0x15+ & \verb+$3 = 0xb+
\\
\hline
\verb+$4 = 0x36+ & \verb+$5 = 0x12+ & \verb+$6 = 0x65+ & \verb+$7 = 0xab+
\\
\hline
\verb+$8 = 0x6+ & \verb+$9 = 0x1e+ & \verb+$10 = 0x2a+ & \verb+$11 = 0xba+
\\
\hline
\verb+$12 = 0xff+ & \verb+$13 = 0x19+ & \verb+$sp = 0x51+ & \verb+$ra = 0x23+
\\
\hline
\verb+PC = 0x50+ & \verb+IR = 0x1230004+ & \verb+TEMP = 0x15+ & \\
\hline
\end{tabular}
\end{center}
\caption{State of registers for Question~\ref{ques:datapath}}
\label{fig:wrampreg}
\end{figure}

\end{enumerate}

\newpage

\begin{table}[h]
\begin{center}
\begin{tabular}{|l|l|p{8cm}|}
\hline
\textbf{Component} & \textbf{Signal Name} & \textbf{Description} \\
\hline
Register File & \texttt{a\_out} & Causes the contents of the
register selected by \texttt{sel\_a} to be output onto the A bus. \\
\cline{2-3}
& \texttt{sel\_a} & Select which register will be output onto the A bus
if \texttt{a\_out} is asserted. \\
\cline{2-3}
& \texttt{b\_out} & Causes the contents of the
register selected by \texttt{sel\_b} to be output onto the B bus. \\
\cline{2-3}
& \texttt{sel\_b} & Select which register will be output onto the B bus
if \texttt{b\_out} is asserted. \\
\cline{2-3}
& \texttt{c\_in} & Causes the value from the C bus to be written
into the register selected by \texttt{sel\_c}.\\
\cline{2-3}
& \texttt{sel\_c} & Select which register to write the value from the C
bus into when the \texttt{c\_in} signal is asserted. \\
\hline
ALU & \texttt{alu\_out} & Causes the result of the current ALU function 
selected by \texttt{alu\_func} to be output to the C bus. \\
\cline{2-3}
& \texttt{alu\_func} & Defines the current operation that the ALU
should perform. ALU functions are defined in table~\ref{table:alu}. \\
\hline
Memory Interface & \texttt{mem\_read} & Causes the contents of the
memory address specified on the A bus to be read and output onto the C
bus. \\
\cline{2-3}
& \texttt{mem\_write} & Causes the value on the B bus to be written
into the memory address specified on the A bus. \\
\hline
Program Counter & \texttt{pc\_out} & Causes the contents of the PC
register to be output onto the A bus. \\
\cline{2-3}
& \texttt{pc\_in} & Causes the value on the C bus to be written into
the PC. \\
\hline
Instruction Register & \texttt{imm\_16\_out} & Causes the least
significant 16 bits of the IR to be output onto the B bus. \\
\cline{2-3}
& \texttt{imm\_20\_out} & Causes the least
significant 20 bits of the IR to be output onto the B bus. \\
\cline{2-3}
& \texttt{sign\_extend} & Causes the output from the IR to be sign
extended to 32bits. \\
\cline{2-3}
& \texttt{ir\_in} & Causes the value on the C bus to be written into
the IR. \\
\hline
Temporary Register & \texttt{temp\_out} & Causes the contents of the
temporary register to be output onto the A bus. \\
\cline{2-3}
& \texttt{temp\_in} & Causes the value on the C bus to be written into
the temporary register. \\
\hline
\end{tabular}
\end{center}
\caption{Descriptions of each of the control signals}
\label{table:signals}
\end{table}
\newpage
\begin{table}[h]
All arithmetic and test/set operations have both signed and unsigned
variants. The unsigned variant is indicated by an operation with a
'u' suffix. A signed variant treats all inputs as signed integers while
the unsigned variant treats inputs as unsigned integers.

\begin{center}
\begin{tabular}{|l|l|l|p{75mm}|}
\hline
\textbf{Type} & \textbf{Name} & \textbf{Function} 
& \textbf{Description} \\
\hline
Arithmetic & \texttt{add, addu} & A + B & Perform an integer
addition between A and B. \\
\cline{2-4}
& \texttt{sub, subu} & A - B & Perform an integer
subtraction between A and B. \\
\cline{2-4}
& \texttt{mult, multu} & A * B & Perform an
integer multiplication between A and B. \\
\cline{2-4}
& \texttt{div, divu} & A / B & Perform an integer
division between A and B. \\
\cline{2-4} 
& \texttt{rem, remu} & A $\bmod$ B & Obtain the remainder from an
integer division between A and B. \\
\hline
Bitwise & \texttt{sll} & A $<<$ B & Shift the value on A left by the number of
places specified by B. Fill with zeros. \\
\cline{2-4}
& \texttt{and} & A AND B & Perform a bitwise AND between A and B. \\
\cline{2-4}
& \texttt{srl} & A $>>$ B & Shift the value on A right by the number of places
specified by B. Fill with zeros. \\
\cline{2-4}
& \texttt{or} & A OR B & Perform a bitwise OR between A and B.\\
\cline{2-4}
& \texttt{sra} & A $>>$ B & Shift the value on A right by the number of places
specified by B. Fill with MSB.\\
\cline{2-4}
& \texttt{xor} & A XOR B & Perform a bitwise XOR between A and B. \\
\hline
Test/set & \texttt{slt, sltu} & out = 1 if ( A $<$ B) & Set out if A
is less than B\\
& & else out = 0 & \\
\cline{2-4}
& \texttt{sgt, sgtu} & out = 1 if ( A $>$ B) & Set out if A
is greater than B \\
& & else out = 0 & \\
\cline{2-4}
& \texttt{sle, sleu} & out = 1 if ( A $\le$ B) & Set out if A
is less than or equal to B\\
& & else out = 0 & \\
\cline{2-4}
& \texttt{sge, sgeu} & out = 1 if ( A $\ge$ B) & Set out if A
is greater than or equal to B\\
& & else out = 0 & \\
\cline{2-4}
& \texttt{seq, sequ} & out = 1 if ( A $=$ B) & Set out if A
is equal to B \\
& & else out = 0 & \\
\cline{2-4}
& \texttt{sne, sneu} & out = 1 if ( A $\neq$ B) & Set out if A
is not equal to B \\
& & else out = 0 & \\
\hline
Misc & \texttt{lhi} & out\tiny$_{[31...16]}$\normalsize~= B\tiny$_{[15...0]}$ &
Set the upper 16 bits of out to be the lower 16 \\
& & out\tiny$_{[15...0]}$\normalsize~= 0 & bits of B. Lower 16 bits of out set to zero. \\ 
\cline{2-4}
& \texttt{inc} & out = A + 1 & Increment A\\
\hline
\end{tabular}
\end{center}
\caption{ALU Operations}
\label{table:alu}
\end{table}


\end{document}
