\documentclass[a4paper,10pt]{article}
\setlength{\parindent}{0ex}
\setlength{\parskip}{1.5ex}
\usepackage[dvips]{graphicx}
\usepackage{epsfig}
\usepackage[a4paper, hmargin=25mm, vmargin=30mm, nohead]{geometry}

\newcommand{\marks}[1]
{\begin{flushright}{\bf (#1 marks)}\end{flushright}}

\begin{document}

\vspace*{-1cm} 

{\centering \large \bf THE UNIVERSITY OF WAIKATO\\}
{\centering \large \bf Department of Computer Science\\[0.5cm]}

{\centering \large \bf COMP201Y Computer Systems 2004 \\}
{\centering \large \bf Mid Year Test - Monday May 31, 2004, 5pm PWC\\[1cm]}
{\centering \large \bf Answer Sheet\\[5mm]}
If you need more space than is provided, write on the reverse of the
page and clearly indicate this.\\[5mm]
Name:\hspace*{5cm}ID Number:\\
\hrule

\section{Multichoice}

\begin{tabular}{p{1.5cm}|p{2cm}|}
\cline{2-2}
1. &  \\
& \\ & \\
\cline{2-2}
2. &  \\
& \\ & \\
\cline{2-2}
3. &  \\
& \\ & \\
\cline{2-2}
4. &  \\
& \\ & \\
\cline{2-2}
5. &  \\
& \\ & \\ 
\cline{2-2}
6. &  \\
& \\ & \\
\cline{2-2}
7. &  \\
& \\ & \\
\cline{2-2}
8. &  \\
& \\ & \\
\cline{2-2}
9. &  \\
& \\ & \\
\cline{2-2}
10. &  \\
& \\ & \\
\cline{2-2}
\end{tabular}

\newpage
\section{Short Answer}
\begin{enumerate}


\item~
\begin{enumerate}

\item~
\vspace{7mm}\hrule\vspace{7mm}\hrule\vspace{7mm}\hrule\vspace{3mm}

\item~
\vspace{7mm}\hrule\vspace{7mm}\hrule\vspace{7mm}\hrule\vspace{3mm}

\item~
\vspace{7mm}\hrule\vspace{7mm}\hrule\vspace{7mm}\hrule\vspace{3mm}

\item~
\vspace{7mm}\hrule\vspace{7mm}\hrule\vspace{7mm}\hrule\vspace{3mm}


\end{enumerate}
%2
\item~

\vspace{7mm}\hrule\vspace{7mm}\hrule\vspace{7mm}\hrule\vspace{7mm}\hrule
\vspace{7mm}\hrule\vspace{3mm}

\newpage
%3
\item~

\vspace{7mm}\hrule\vspace{7mm}\hrule\vspace{7mm}\hrule\vspace{7mm}\hrule
\vspace{7mm}\hrule\vspace{7mm}\hrule\vspace{7mm}\hrule\vspace{7mm}\hrule\vspace{7mm}\hrule\vspace{3mm}
%4
\item~

\vspace{7mm}\hrule\vspace{7mm}\hrule\vspace{7mm}\hrule\vspace{7mm}\hrule
\vspace{7mm}\hrule\vspace{3mm}

%5
\item~

\vspace{7mm}\hrule\vspace{7mm}\hrule\vspace{7mm}\hrule\vspace{7mm}\hrule
\vspace{7mm}\hrule\vspace{7mm}\hrule\vspace{7mm}\hrule\vspace{7mm}\hrule
\vspace{7mm}\hrule\vspace{7mm}\hrule\vspace{7mm}\hrule\vspace{3mm}

\newpage
%6
\item~

\vspace{7mm}\hrule\vspace{7mm}\hrule\vspace{7mm}\hrule
\vspace{7mm}\hrule\vspace{7mm}\hrule\vspace{7mm}\hrule\vspace{7mm}\hrule
\vspace{7mm}\hrule\vspace{7mm}\hrule\vspace{7mm}\hrule\vspace{7mm}\hrule\vspace{3mm}
%7
\item~

\vspace{7mm}\hrule\vspace{7mm}\hrule\vspace{7mm}\hrule\vspace{7mm}\hrule
\vspace{7mm}\hrule\vspace{7mm}\hrule\vspace{7mm}\hrule\vspace{7mm}\hrule
\vspace{7mm}\hrule\vspace{3mm}

%8
\item~

\vspace{7mm}\hrule\vspace{7mm}\hrule\vspace{7mm}\hrule\vspace{7mm}\hrule
\vspace{7mm}\hrule\vspace{7mm}\hrule\vspace{7mm}\hrule\vspace{7mm}\hrule
\vspace{7mm}\hrule\vspace{3mm}
%9
\item~

\vspace{7mm}\hrule\vspace{7mm}\hrule\vspace{7mm}\hrule\vspace{7mm}\hrule
\vspace{7mm}\hrule\vspace{7mm}\hrule\vspace{7mm}\hrule\vspace{3mm}
%10
\item~

\vspace{7mm}\hrule\vspace{7mm}\hrule\vspace{7mm}\hrule\vspace{7mm}\hrule
\vspace{7mm}\hrule\vspace{7mm}\hrule\vspace{7mm}\hrule\vspace{3mm}
%11
\item~

\vspace{7mm}\hrule\vspace{7mm}\hrule\vspace{7mm}\hrule\vspace{7mm}\hrule
\vspace{7mm}\hrule\vspace{3mm}

%\newpage
%12
\item~

\vspace{7mm}\hrule\vspace{7mm}\hrule\vspace{7mm}\hrule\vspace{7mm}\hrule\vspace{7mm}\hrule
\vspace{7mm}\hrule\vspace{7mm}\hrule\vspace{7mm}\hrule\vspace{3mm}
%13
\item~

\vspace{7mm}\hrule\vspace{7mm}\hrule\vspace{7mm}\hrule\vspace{7mm}\hrule\vspace{7mm}\hrule
\vspace{7mm}\hrule\vspace{7mm}\hrule\vspace{7mm}\hrule\vspace{3mm}
%14
\item~

\vspace{7mm}\hrule\vspace{7mm}\hrule\vspace{7mm}\hrule\vspace{7mm}\hrule\vspace{7mm}\hrule
\vspace{7mm}\hrule\vspace{7mm}\hrule\vspace{7mm}\hrule\vspace{7mm}\hrule
\vspace{7mm}\hrule\vspace{7mm}\hrule\vspace{7mm}\hrule\vspace{3mm}

\newpage
%15
\item

\begin{enumerate}
 \item What are the contents of the following registers each time the
breakpoint is encountered? You should fill in one line each time you
believe a breakpoint was hit. You should not need more lines than are
provided, although you may need less.

\begin{center}
\begin{tabular}{|c|c|c|c|}
\hline\hspace{8mm}\textbf{\$2}\hspace{8mm} & \hspace{8mm}\textbf{\$3}\hspace{8mm} & \hspace{8mm}\textbf{\$4}\hspace{8mm} & \hspace{8mm}\textbf{\$5}\hspace{8mm} \\
\hline & & & \\ & & & \\
\hline & & & \\ & & & \\
\hline & & & \\ & & & \\
\hline & & & \\ & & & \\
\hline & & & \\ & & & \\
\hline & & & \\ & & & \\
\hline & & & \\ & & & \\
\hline

\end{tabular}
\end{center}

\item What does this program do?

\vspace{7mm}\hrule\vspace{7mm}\hrule\vspace{7mm}\hrule\vspace{7mm}\hrule
\vspace{7mm}\hrule\vspace{7mm}\hrule\vspace{3mm}

\end{enumerate}

\newpage
%16
\item~
\begin{enumerate}
\item Indicate which lines of WRAMP assembler were generated for the given C code.

\begin{enumerate}
\item~
\vspace{7mm}\hrule\vspace{3mm}
\item~
\vspace{7mm}\hrule\vspace{3mm}

\end{enumerate}

\item Draw a diagram to show the stack frame created by the function
\texttt{fact}.

\begin{center}
 \psfig{figure=stack.eps, width=\linewidth}
\end{center}

\end{enumerate}

\newpage
%17
\item~
\begin{enumerate}

\item Label the WRAMP datapath.
\vspace{1cm}

\begin{figure}[h]
\begin{center}
    \psfig{figure=datapath_ul.eps, width=.85\linewidth}
     \label{fig:wrampblok}
  \end{center}
\end{figure}
\vspace{1cm}

\item~

\vspace{7mm}\hrule
\vspace{7mm}\hrule\vspace{7mm}\hrule\vspace{7mm}\hrule\vspace{7mm}\hrule
\vspace{7mm}\hrule\vspace{7mm}\hrule\vspace{7mm}\hrule\vspace{7mm}\hrule
\vspace{7mm}\hrule\vspace{7mm}\hrule\vspace{3mm}

\item~
\vspace{7mm}\hrule\vspace{7mm}\hrule\vspace{3mm}


\end{enumerate}

\end{enumerate}

\end{document}




