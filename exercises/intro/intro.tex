\documentclass[a4paper,10pt]{article}
\setlength{\parindent}{0ex}
\setlength{\parskip}{1.5ex}
\usepackage[dvips]{graphicx}
\usepackage{epsfig}
\usepackage[a4paper, hmargin=25mm, vmargin=30mm, nohead]{geometry}

\usepackage{fancyhdr}
\fancypagestyle{rcsfooters}{
\fancyfoot[L]{\small $ $RCSfile: intro.tex,v $ $}
\fancyfoot[R]{\small $ $Revision: 1.3 $ $}
}

\renewcommand{\headrulewidth}{0pt}

%INCLUDE OUR GLOBALS
\usepackage{rex201}
\usepackage{styles201}
\usepackage{ex201}


\begin{document}

\EXHEADING{\INTRONO}{\INTROTITLE}{%
Verification Date:~\INTRODUE\\
Excercise Test Date:~\CWRAMPTESTDATE\\
}


\section{Assessment}
To show that you have completed this exercise to a satisfactory
standard you will be required to demonstrate at a verification session
that the program you wrote in \textbf{Question~\ref{ques:final}} 
of this exercise
runs correctly. In addition the marker may ask one or two questions so
they establish that you understand what you have done.  These
verification sessions will be held in Lab 1 during the following two
times:

\begin{itemize}
\item \INTRODUE~\MORNINGASSESS
\item \INTRODUE~\AFTERNOONASSESS 
\end{itemize}

In an attempt to ensure that the verification process runs smoothly
six machines in \ASSESSROOM\ will be booked for this purpose during the two
sessions. When it is your turn to have your program verified you
should log onto one of these machines and get your program ready to
run on the REX board. A marker will then verify your program. Once
your program has been verified you should log off immediately to allow
the next person to prepare to have their program verified.

In addition to the test there will be a closed book excercise test on
\textbf{\CWRAMPTEST}.  The test will cover material 
from excercises 2, 3 and 4.

The tutorial times are listed on the course webpage:

\begin{center}
\src{\WEBPAGEBASE}
\end{center}

\section{Ojectives}
The objective of this exercise is to familiarise 
yourself with writing, assembling, linking, running, and
debugging assembly code for the WRAMP processor and the REX board.

Before attempting this excercise you should have read the 
chapter `Introduction to Rex and WRAMP' in your course manual.  The material covered in this chapter of the manual \textbf{is examinable}.

\section{Questions}
\begin{enumerate}
\item\label{ques:first} Write a program which continually (i.e. is in
an infinite loop) reads the values on the switches and outputs the
value read to the two seven segment displays.  For example if the
switches are set to the pattern shown in Figure \ref{fig:switches} the
two seven segment displays should show \src{92}.

\item\label{ques:two} Modify your program in Question \ref{ques:first}
to continually output the number of switches currently set (in the
down position or one) to the seven segment display. For example if the
switches are set to the pattern shown in Figure \ref{fig:switches}
then the seven segment display should display \src{03}.

\item\label{ques:final} Modify the program written in Question
\ref{ques:two} to encrypt the count of set switches before outputting
it to the seven-segment display. The encryption algorithm to be
applied is the very simple mapping function defined in Table
\ref{table:encode}. For example, the value \src{4} encrypted would be
displayed as \src{05}, and the value \src{8} encrypted would be
displayed as \src{07}.
\end{enumerate}

\begin{figure}[!hb]
\begin{center}
%\begin{tabular}{|c|}
%\hline
\includegraphics[width=15cm]{switches.eps}
%\\
%\hline
%\end{tabular}
\caption{Example}
\label{fig:switches}
\end{center}
\end{figure}


\begin{table}
\begin{center}
\begin{tabular}{|c|c|}
\hline
\verb|       |0\verb|       | & \verb|       |1\verb|       | \\ \hline
1 & 0 \\ \hline
2 & 8 \\ \hline
3 & 2 \\ \hline
4 & 5 \\ \hline
5 & 4 \\ \hline
6 & 9 \\ \hline
7 & 3 \\ \hline
8 & 7 \\ 
\hline
\end{tabular}
\end{center}
\label{table:encode}
\caption{Mapping Function for Encryption Process}
\end{table}






\thispagestyle{rcsfooters}
\pagestyle{rcsfooters}
\end{document}