\documentclass[a4paper,10pt]{article}
\setlength{\parindent}{0ex}
\setlength{\parskip}{1.5ex}
\usepackage[dvips]{graphicx}
\usepackage{epsfig}
\usepackage[a4paper, hmargin=25mm, vmargin=30mm, nohead]{geometry}

\usepackage{fancyhdr}
\fancypagestyle{rcsfooters}{
\fancyfoot[L]{\small $ $RCSfile: kernel.tex,v $ $}
\fancyfoot[R]{\small $ $Revision: 1.3 $ $}
}

%INCLUDE OUR GLOBALS
\usepackage{rex201}
\usepackage{styles201}
\usepackage{ex201}

\renewcommand{\headrulewidth}{0pt}

\begin{document}

\EXHEADING{\KERNNO}{\KERNTITLE}{%
Verification Date:~\KERNDUE\\
Exercise Test Date:~\TESTTHREEDATE \\
}

\section{Assessment}

To show that you have completed this exercise to a satisfactory
standard you will be required to demonstrate at a verification session
that the program you wrote in \textbf{Question~\ref{ques:final}} of
this exercise runs correctly. In addition the marker may ask one or
two questions so they establish that you understand what you have
done.  These verification sessions will be held in \ASSESSROOM\ during the
following two times:

\begin{itemize}
\item \KERNDUE~\MORNINGASSESS
\item \KERNDUE~~\AFTERNOONASSESS
\end{itemize}

In an attempt to ensure that the verification process runs smoothly
six machines in \ASSESSROOM\ will be booked for this purpose during the two
sessions. When it is your turn to have your program verified you
should log onto one of these machines and get your program ready to
run on the REX board. A marker will then verify your program. Once
your program has been verified you should log off immediately to allow
the next person to prepare to have their program verified.

In addition to the verification process there will be a closed book
exercise test on~\textbf{\TESTTHREE}. The test can cover any material
from this exercise.

\section{Objective}

The aim of this exercise is to develop your understanding of
multitasking through implementation of a simple multitasking kernel
for the REX board.

\section{Introduction}

This exercise continues from the previous simulation exercise. In this
exercise you will implement a simple multitasking kernel, similar to
the one shown in the simulator.

This exercise builds from both the ``\INTERTITLE'' exercise as well as
the ``\MTKSTITLE'' exercise. The exercise assumes that you will use
the I/O Device, Exception and Stack material in the REX/WRAMP manual,
as well as the course lecture material to gain the knowledge required
to complete it successfully.

Your kernel will include:
\begin{itemize}
\setlength{\itemsep}{0cm}
\setlength{\parskip}{0cm}
\item An exception handler
\item A dispatcher
\item Some simple jobs
\end{itemize}

By the end of this exercise your kernel must implement the following
features which have been discussed in the lecture notes.
\begin{itemize}
\setlength{\itemsep}{0cm}
\setlength{\parskip}{0cm}
\item Saving and restoring the complete context of the jobs.
\item Individual stacks for each job
\item A PCB linked list for job selection
\end{itemize}

\section{Questions}

\begin{enumerate}
\item Write a program which uses the REX timer to produce an interrupt
each second. Inside your exception handler increment a memory location
which contains a count of the number of seconds the program has been
running.

Inside the main-loop of your code write a job which continuously loads
this counter and displays it to the Linux machine as a time in minutes
and seconds. Write a compliant WRAMP subroutine which accepts a single
argument, and prints that argument to the Linux machine. Use this
subroutine to print each of the characters that make up the time. Make
sure that your subroutine follows all the requirements provided in the
WRAMP Stack Guide.

\item
\label{ques:parta}

Modify this program so that the timer interrupts 100 times per
second. Alter the code which increments the seconds counter so that
the counter is still only incremented once per second.

\item Add to your code a dispatcher and a single PCB. Make the PCB
link field point to itself. Your dispatcher should save the state of
the current job, and then load the state of the next job. As there is
only a single job your dispatcher will end up restoring the same
state.

To complete this section you will need to introduce a current job
pointer, as well as correctly initialising a PCB. You should give your
job a timeslice of 2, that is you will call the dispatcher once for
every 2 timer interrupts.

The code that formed the main-loop of the original program is now the
only job being run by a multitasking kernel. You should make sure that
you enter the job initially by jumping to an entry point inside your
dispatcher.

This question is the hardest of the exercise and should be completed
carefully. Make regular backups of working code before you add more
features.

\item Add a second job to your program. This job should read the value
on the switches and print the value to the seven segment display.

To add this job you will need to add a second PCB, correctly
initialise it, and make sure it is part of the PCB linked list.

As you now have two jobs you will also need to create a separate stack
for each of the jobs. Follow the examples provided in the lecture
notes on how to do this. You can make the stacks of any size, but
serious problems can occur if the stack is too small. For the jobs you
will be running we recommend a stack size of 100 words for each task.

Make sure that both jobs are using the same registers in order to
check that the contents of these registers are being correctly
preserved by your dispatcher.

To see the way your kernel is operating you may wish to slow the
timer down. If the timer interrupts only once per second you should
see that each job runs for two seconds before the other starts
running.

\item
\label{ques:final}

Your kernel should now be running each job inside its own individual
environment. If this is the case then there should be no reason that
you would have to write each of the jobs you run. If you are correctly
saving all the context of a job and restoring it properly then a job
should never know that it is running alongside other code.

To test this we have provided some code which will make up the final
job of your multitasking kernel. This job is a very simple game which
is played on the terminal attached to the REX board.

Add a third PCB and stack to your kernel. The provided job starts at a
label called \src{gamejob}. The game is provided in a separate
object file called \filename{game.o}. This file is provided in the
\filename{/home/201/ex8/} directory. Do \textbf{not} copy this file to
your home account. Instead link directly against this file.

If you are having problems remembering how to link multiple object
files together, refer back to the early exercises where your code was
linked against the \filename{lib.o} object.

For the curious, the source to this game is provided in the
\filename{game.s} file in the same directory. You are welcome to take a
copy of this code to play with, but for the final verification your
kernel must run successfully when linked against the \filename{game.o}
file provided.

\end{enumerate}
\thispagestyle{rcsfooters}
\pagestyle{rcsfooters}

\end{document}






