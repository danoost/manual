\documentclass[a4paper,10pt]{article}
\setlength{\parindent}{0ex}
\setlength{\parskip}{1.5ex}

\usepackage[dvips]{graphicx}
\usepackage{epsfig}
\usepackage[a4paper, hmargin=25mm, vmargin=30mm, nohead]{geometry}

\usepackage{fancyhdr}
\fancypagestyle{rcsfooters}{
\fancyfoot[L]{\small $ $RCSfile: io.tex,v $ $}
\fancyfoot[R]{\small $ $Revision: 1.1 $ $}
}

%INCLUDE OUR GLOBALS
\usepackage{rex201}
\usepackage{styles201}
\usepackage{ex201}

\begin{document}

\EXHEADING{\IONO}{\IOTITLE}{%
Verification Date:~\IODUE\\
Exercise Test Date:~\IOTEST\\  	
}



\section{Assessment}

To show that you have completed this exercise to a satisfactory
standard you will be required to demonstrate at a verification session
that the program you wrote in \textbf{Question~\ref{ques:final}} of this exercise
runs correctly. In addition the marker may ask one or two questions so
they establish that you understand what you have done.  These
verification sessions will be held in \ASSESSROOM during the following two
times: 

\begin{itemize}
\item \IODUE~\MORNINGASSESS
\item \IODUE~\AFTERNOONASSESS
\end{itemize}

In an attempt to ensure that the verification process runs smoothly
six machines in \ASSESSROOM will be booked for this purpose during the two
sessions. When it is your turn to have your program verified you
should log onto one of these machines and get your program ready to
run on the REX board. A marker will then verify your program. Once
your program has been verified you should log off immediately to allow
the next person to prepare to have their program verified.

In addition to the verification process there will be a closed book
exercise test on~\textbf{\IOTEST}. The test can cover any material
from exercises 5 or 6.

\section{Objective}

The purpose of this exercise is to reinforce the ideas presented in
lectures on I/O devices. To achieve this goal you will be asked to
write WRAMP code to correctly talk to the I/O devices provided on the
REX board.

\section{Introduction}
There are a range of I/O systems provided on the REX Board are. This
exercise will deal with two particular devices:

\begin{itemize}
\item A parallel port
\item A dual serial port
\end{itemize}

\subsection{Parallel Port}
\label{sec:parallel}

The REX board parallel port is connected to the switches and the pair
of seven segment display devices. To obtain the current value set on
the switches you should read a word from the location~\LOCSWS. The
lower 8 bits of this word will be set to the current value on the
switches. The upper 24 bits will be set to zero.

The first seven segment display can be written to by writing to the
address~\LOCSSDONE~and the second display by writing to~\LOCSSDTWO. By
default writing to these locations will decode the least significant
four bits and display it as a hexadecimal value to the display. The
upper 28 bits are ignored.


\subsection{Serial Port}


The REX board provides two serial interfaces, one of which is
connected to the Linux machine and the other is connected to the
terminal that should be sitting on the shelf above the board. 

For each of the serial interfaces there are 4 registers accessible to
the CPU. These registers are the transmit data register, the receive
data register, the status register and the control register.

The transmit data register holds the value that is to be sent out of
the serial port. The receive data register holds the value that has
been received in the serial port. The status register indicates if
there is a value in the receive data register and also if the value in
the transmit data register has been sent. The control register allows
the configuration of the serial port. For this exercise the control
register will have already been configured for you by the monitor so
it will not need to be altered. Each serial port has a base address
and the 4 registers for each serial port are expressed as an offset
from this address. The base addresses for each port are defined in
table~\ref{table:serialbase} and the offsets are defined in
table~\ref{table:serialoffset}. The format of the status register for
each of the serial devices is shown in table~\ref{table:serialstat}.

\begin{table}[h]
\begin{center}
\begin{tabular}{|l|l|}
\hline
\textbf{Port} & \textbf{Base Address} \\
\hline
Linux machine & \LOCLINBASE \\ 
\hline
Terminal & \LOCTERMBASE \\
\hline
\end{tabular}
\end{center}
\caption{Base addresses for each serial port}
\label{table:serialbase}
\end{table}

\begin{table}[h]
\begin{center}
\begin{tabular}{|l|l|}
\hline
\textbf{Register} & \textbf{Address} \\
\hline
Transmit Data & Base + \LOCTRANOFFSET \\ 
\hline
Receive Data & Base + \LOCRECOFFSET \\
\hline
Control & Base + \LOCCONTROLOFFSET \\
\hline
Status & Base + \LOCSTATOFFSET \\
\hline
\end{tabular}
\end{center}
\caption{Offsets for each register}
\label{table:serialoffset}
\end{table}

\begin{table}[h]
\begin{center}
\begin{tabular}{|l|l|}
\hline
\textbf{Bit} & \textbf{Function} \\
\hline
0 & Receive Data Ready \\
 & 1 if data in receive data register, 0 otherwise \\
\hline
1 & Transmit Data Sent \\
 & 1 if ready for next character, 0 if data is still being transmitted  \\
\hline
\end{tabular}
\end{center}
\caption{Status register format}
\label{table:serialstat}
\end{table}

\section{Questions}

\begin{enumerate}

\item Write a program which continually (i.e. is in an infinite loop)
reads the values on the switches and outputs the value read to the two
seven segment displays. This is the same question as question 10 of
exercise 3, however this time you should \textbf{not} use the writessd
and readswitches routines that you were provided with for that
exercise. You should instead directly talk to the parallel port using
the addresses supplied in section~\ref{sec:parallel}.

\item 
\label{ques:single}

Write a program that sends an X to the serial port that is
connected to the terminal above the board. Do not use any library
routines to do the I/O (e.g. putstr). Instead manipulate the serial
port directly.

\item 
\label{ques:badmult}
Modify the program you wrote in question~\ref{ques:single} to output
25 Xs' without first checking the status register to make sure that
the serial device is ready to send the next character. When you run
the program what happens?  Why? \textit{Note:} single stepping through
your program will cause very different effects to running your program
without any breaks in execution. Why?

\item Modify the program you wrote in question~\ref{ques:badmult} so
that all of the Xs' are sent correctly out through the serial port.

\item 
\label{ques:upperlower}
Write a program that continually reads characters from the serial port
connected to the terminal, converts any \textbf{lowercase} characters
to \textbf{uppercase} and outputs them to the serial port connected to
the Linux machine. (e.g. if an 'a' is received on the serial port
connected to the terminal then an 'A' will be output to the serial
port connected to the Linux machine. However, if an 'A' is received on
the serial port connected to the terminal then an 'A' is output to the
serial port connected to the Linux machine.)

\item 
\label{ques:switchcase}
Modify the program you wrote in question~\ref{ques:upperlower} so that
it reads the characters from the Linux machine and outputs to the
terminal. Also modify the program so that instead of making all
characters uppercase it changes the case of all characters
input. (i.e. you should make all lowercase characters uppercase and
all uppercase characters lower case.) You should be careful not to
modify any non-character that is input such as spaces or punctuation.

\item 
\label{ques:final}
Write a program that combines the functionality of both programs
in~\ref{ques:upperlower} and~\ref{ques:switchcase}
simultaneously. That is the program should both output
\textbf{uppercase} versions of the characters typed on the terminal to
the Linux machine and switch the case of all characters input on the
Linux machine and display that to the terminal. To achieve this your
program will to contain a loop that repeatedly polls each of the
devices (i.e. the terminal and the Linux machine) in turn to see if a
character has been received, and if so outputs it to the appropriate
serial interface.

\end{enumerate}

\thispagestyle{rcsfooters}
\pagestyle{rcsfooters}
\end{document}