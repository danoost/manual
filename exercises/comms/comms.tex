\documentclass[a4paper,10pt]{article}
\setlength{\parindent}{0ex}
\setlength{\parskip}{1.5ex}
\usepackage[dvips]{graphicx}
\usepackage{epsfig}
\usepackage[a4paper, hmargin=25mm, vmargin=30mm, nohead]{geometry}

\usepackage{fancyhdr}
\fancypagestyle{rcsfooters}{
\fancyfoot[L]{\small $ $RCSfile: comms.tex,v $ $}
\fancyfoot[R]{\small $ $Revision: 1.1 $ $}
}

\renewcommand{\headrulewidth}{0pt}

%INCLUDE OUR GLOBALS
\usepackage{rex201}
\usepackage{styles201}
\usepackage{ex201}


\begin{document}

\EXHEADING{\COMMSNO}{\COMMSTITLE}{Final Verification Date:~\COMMSDUE\\}


\section{Assessment}

This exercise is electronically marked. Once your code passes the
electronic marking you will receive an e-mail confirmation sent to your
waikato e-mail address. The e-mail will be addresses to:

\begin{center}
\texttt{<your username>@waikato.ac.nz}
\end{center}

If you cannot collect mail sent to this address you should contact the
course tutors at:

\begin{center}
\texttt{contact-cs201@cs.waikato.ac.nz}
\end{center}

Once you have received the conformation e-mail you should submit your code
using the following online submission system:

\begin{center}
\verb+http://byerley.cs.waikato.ac.nz/~tonym/submit/+\\
\end{center}

If you do not receive a conformation e-mail and you believe your code is
operating correctly you should contact the course tutors at the above
e-mail address.

To avoid a test late in the semester there is no class test for this
exercise.  The material from this exercise may be tested in the final
exam. (This is also true for the other exercises).

\section{Objectives}

The aim of this exercise is to investigate simple error detection
mechanisms. To achieve this you will write two programs.  The first adds
error checking information to a block of data and transmits the frame.  The
second receives the frame and checks for errors.  You are not required to
correct the errors, just to report whether an error has occurred.

\section{Introduction} 

The programs you write for this exercise will be linked with routines which
supply the data and provide (error prone) transmission between your
programs.  Another routine allows you to check whether you have correctly
detected any errors that occur.

\newpage
The main lines of your programs should look something like this:

{\bf Transmit program}

\begin{verbatim}
#include "/home/201/ex10/libcomms.h"

connect_transmitter();

while ( NULL != (raw = gen()) )
   {
   encode(raw, coded);
   transmit_frame(coded);
   }
\end{verbatim}

{\bf Receiver program}

\begin{verbatim}
#include "/home/201/ex10/libcomms.h"

connect_receiver();

while ( -1 != receive_frame(coded) )
   {
   ok = res(check(coded));
   }	
\end{verbatim}

You will need to write the \src{encode} and \src{check} routines.

The following figure shows the structure of the complete system.

\begin{center}
\psfig{figure=comms_block.eps}
\end{center}

\newpage
\section{The Supplied Routines}

The six supplied routines are in the library file
\filename{/home/201/ex10/libcomms.a}.  When you compile your program you
should include this file as a parameter to \program{g++}.  There is a
\filename{/home/201/ex10/libcomms.h} file which contains prototypes for the
supplied routines.

If you wish to do this exercise on equipment other than the lab you
will need a copy of this library.  However {\em do not take a copy of
this library without talking to the course tutors first.}

The following table shows which program you will use each routine in:
\vspace{1cm}


\centerline{\begin{tabular}{|l|l|}
\hline
Routine & Used in \\
\hline
\src{gen()}				& Transmitter \\
\src{connect\_transmitter()} 	& Transmitter \\
\src{transmit\_frame()}		& Transmitter \\
\src{connect\_receiver()}       	& Receiver    \\
\src{receive\_frame()}		& Receiver    \\
\src{res()}				& Receiver    \\
\hline
\end{tabular}
}

\subsection{\texttt{gen()}}

Usage:

\hspace*{1cm} \verb+char *gen(void)+ 

\verb+gen()+ returns a string.  The string contains a
sequence of ``1'' and ``0'' characters, representing the bits to be
transmitted.  That is, for the sake of this exercise, {\em each bit is
represented by a whole ASCII character}.

Generated strings will not exceed 100 bits. The length of the generated
string will always be a multiple of 8.

When there is no more data \verb+gen()+ returns a null string.

\subsection{\texttt{connect\_transmitter()}}

Usage:

\hspace*{1cm} \verb+void connect_transmitter(void)+ 

This routine establishes the connection and must be called before the
\verb+transmit_frame()+ routine and also before the
\verb+connect_receiver()+ routine.  This means you must start your
transmitter program before you start you receiver program.

\subsection{\texttt{connect\_receiver()}}

Usage:

\hspace*{1cm} \verb+void connect_receiver(void)+ 

This routine opens the connection for receiving data and must be called
before the \verb+receive_frame()+ routine can be used.  It must be
called after the \verb+connect_transmitter()+ routine.

\subsection{\texttt{transmit\_frame()}}

Usage:

\hspace*{1cm} \verb+void transmit_frame(char *)+ 

\verb+transmit_frame()+ takes a single parameter that is a string.
The string should contain a sequence of ``1'' and ``0'' characters
like the string returned from \verb+gen()+.  No other characters
should be included.

\verb+transmit_frame()+ simulates transmission of the data including
introducing some errors.  

\subsection{\texttt{receive\_frame()}}

Usage:

\hspace*{1cm} \verb+int receive_frame(char *)+ 

\verb+receive_frame+ takes a single parameter that is a string and
writes into the string a sequence of ``1'' and ``0'' characters
representing the result of transmission of the data.

\verb+receive_frame+ normally returns a positive number but returns
\verb+-1+ when there are no more frames to process.

Note: This routine does not allocate storage for the received frame.
The parameter must point to a buffer that is sufficiently large to
receive any message that will be sent from \verb+transmit_frame()+.

\subsection{\texttt{res()}}

Usage:

\hspace*{1cm} \verb+int res(int)+ 


\verb+res()+ treats its integer parameter as a boolean specifying
whether the data was transmitted correctly.  If you detect an error
call \verb+res(0)+.  If you do not detect an error call \verb+res(1)+.

\verb+res()+ returns 1 if you provide the correct parameter and 0 otherwise.
\verb+res()+ also writes a log file (\verb+res_log+).

You {\em must} call \verb+res()+ for each block generated and
transmitted.  \verb+res()+ will be used as part of the assessment
process.


\section{What to Do}

\begin{enumerate}
\item Use parity to detect errors.  Add one parity bit for every 8
      bits of data.  This is the minimum requirement.  If you have
      correctly implemented parity (and only parity) you will receive
      email that you have met the minimum requirement when you first
      run your program in the lab with the library provided.  If you
      do not get the email and are sure you have implemented parity,
      contact a tutor.

      {\em When you have received the email use the submit program\\
      (\verb+http://byerley.cs.waikato.ac.nz/~tonym/submit/+) to
      submit your code so that we have a record of your program.}

\item Add either longitudinal parity or a checksum based bcc.

\item Once you have this working correctly try a more sophisticated
      algorithm that catches as many errors as possible.  

\end{enumerate}

{\bf Notes:}

\begin{enumerate}
\item
The messages in this exercise are supplied and sent as one character
for each bit.  That is an ASCII string of characters is used but only
two character values are valid, ``1'' and ``0''.  This makes the
coding of the exercise simpler but would be very inefficient in a
real communications protocol.

Although there is no structure in the bit stream you can assume that
every 8 bits represents a character in the original data stream.

\item \vspace{3mm}

{\em DO NOT} modify the string returned by \verb+gen()+.  You do not
know how much space is allocated to this string and can not extend it
in any way.  Instead allocate space for a new string and copy
appropriate bits into the new string.

\item
To allow easy debugging the data and errors generated by the system
will be the same from run to run. 

\item
For the sake of this exercise the error rate is much higher than is
normal on communications lines.

\item
Run your two programs in separate xterm windows on the same
workstation.  You must be in the same directory when you run both
programs.
\end{enumerate}

\section{Extension}

Two possible extensions include error detection that is more robust
and forward error correction.  If you use forward error detection the
\verb+res()+ function in the library will not be useful.  Contact a
tutor if you hare having difficulties discovering which transmissions
have errors.

\thispagestyle{rcsfooters}
\pagestyle{rcsfooters}

\end{document}

