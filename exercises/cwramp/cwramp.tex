\documentclass[a4paper,10pt]{article}
\setlength{\parindent}{0ex}
\setlength{\parskip}{1.5ex}
%\usepackage[dvips]{graphicx}
\usepackage{epsfig}
\usepackage{verbatim}
\usepackage[a4paper, hmargin=25mm, vmargin=30mm, nohead]{geometry}

\usepackage{fancyhdr}
\fancypagestyle{rcsfooters}{
\fancyfoot[L]{\small $ $RCSfile: cwramp.tex,v $ $}
\fancyfoot[R]{\small $ $Revision: 1.7 $ $}
}

%INCLUDE OUR GLOBALS
\usepackage{rex201}
\usepackage{styles201}
\usepackage{ex201}


\renewcommand{\headrulewidth}{0pt}

\begin{document}

\EXHEADING{\CWRAMPNO}{\CWRAMPTITLE}{%
Verification Date:~\CWRAMPDUE\\
Exercise Test Date:~\TESTONEDATE\\
}

\section{Assesment}
To show that you have completed this exercise to a satisfactory
standard you will be required to demonstrate at a verification session
that the program you wrote in \textbf{Question~%\ref{ques:final}} 
2}
of this exercise
runs correctly. In addition the marker may ask one or two questions so
they establish that you understand what you have done.  These
verification sessions will be held in \ASSESSROOM\ during the following two
times:

\begin{itemize}
\item \CWRAMPDUE~\MORNINGASSESS
\item \CWRAMPDUE~\AFTERNOONASSESS
\end{itemize}

In an attempt to ensure that the verification process runs smoothly
six machines in \ASSESSROOM\ will be booked for this purpose during the two
sessions. When it is your turn to have your program verified you
should log onto one of these machines and get your program ready to
run on the REX board. A marker will then verify your program. Once
your program has been verified you should log off immediately to allow
the next person to prepare to have their program verified.

In addition to the test there will be a closed book exercise test on
\textbf{\TESTONE}.  The test will cover material 
from exercises 1, 2 and 3.


\section{Objective}
The objective of this assignment is to gain some insight 
into the relationship between high level
programming languages and assembly level languages, and 
the importance of the stack in the
implementation of functions. To achieve this you will 
write a function in both C and WRAMP assembly
that will be compiled, assembled and executed on the 
REX boards.

\section{Questions}
\subsection{Question 1} \label{ques:first}
\emph{%
Write a C function called \src{counter} that takes two integer 
arguments, called \src{value} and \src{direction}. The function 
prototype would appear as follows:
}
\begin{center}
\begin{verbatim}
     void counter(int value, int direction)
\end{verbatim}
\end{center}

This function has two modes of operation. If \src{direction} 
is set to zero, the seven segment display
should count from zero up to \src{value}. If \src{direction} 
is set to one, the seven segment display should
count down from \src{value} to zero.

In order to be able to see each individual value written to 
the display a delay must be inserted between
each write.
To write to the seven segment display you should 
use a function called writessd, similar to the one
provided in exercise three.    This function is 
provided in a file called \filename{/home/201/lib.o}.    The
function prototypes for this library are declared in the 
file \filename{/home/201/lib.h}. Your C code must
include this header. You should have something 
like the following at the top of your C function:


\begin{verbatim}
     #include "/home/201/lib.h"
\end{verbatim}

(note the `\#' should be hard up against the left margin.)

\filename{/home/201/lib.o} also provides the delay 
function. You should call this function once (with no
arguments) in between writing each value to the seven segment display.
You must name your function counter, and the 
C file should not contain any other function.
Specifically you should not include a \src{main( )} function. 
The \src{main()} function will be provided and is
detailed later.
A full list of the functions provided by the 
library and the arguments each of these expect is detailed in
Appendix %ref.
You should call the file that you create \filename{counter.c} 
and save it in a directory for this exercise.
Compile your C program using \program{wcc} (a C 
compiler that generates code for the WRAMP processor) and
examine the resulting WRAMP code (\filename{counter.s}). You should 
pay particular attention to examining
the relationship between the statements in the 
C program and the resulting WRAMP assembler
statements as you will be tested on your understanding 
of these relationships in the exercise test. The
command to compile your program into WRAMP code using wcc is:
\begin{verbatim}
     wcc -�S counter.c
\end{verbatim}

Assemble the WRAMP code using the following command:
\begin{verbatim}
     wasm counter.s
\end{verbatim}

If both of these commands completed without errors you 
should now have a file called \filename{counter.o} in
the current directory.
To allow you to test your function on a REX board 
two functions are provided. The first function,
\src{main()}, provides the entry point for the code.    
This function is provided in the file
\filename{/home/201/ex3/main.c}. The function 
simply reads the current value on the switches then calls a
subroutine called \src{counter\_ctrl}. \src{counter\_ctrl} 
is the second function provided and is declared
in the file \filename{/home/210/ex3/ctrl.c}. This 
function accepts a single argument, which is the value on
the switches. From this value the function 
determines the direction and value that your counter
function will be called with. The provided 
\src{counter\_ctrl} simply uses the most significant bit (bit 7)
as direction, and the remaining bits (bits 0 to 6) 
as value. You should open and view both of
these files to ensure you understand how they work.
You do not need to copy these files to your 
home directory as precompiled versions of both of these are
given in \filename{/home/201/ex3/main.o} and 
\filename{/home/201/ex3/ctrl.o} respectively.
To create an uploadable file that includes the 
compiled version of your C function counter run the
following link command on one line:

\begin{verbatim}
     wlink �-o ex3_1.srec counter.o /home/201/ex3/main.o 
              /home/201/ex3/ctrl.o /home/201/lib.o
\end{verbatim}

This links together your \src{counter} function and the provided 
\src{main}, \src{counter\_ctrl} and library
functions into one uploadable file called \filename{ex3\_1.srec}.
You should upload this file to the REX board and confirm 
that the program operates correctly.

The second stage of this assignment is to replace
 the provided \src{counter\_ctrl} function with one that
you write in assembly. For C code and 
handwritten assembly code to be used interchangeably in the
same program both the compiler and assembly coder 
must follow strict conventions on how the two
interact. These conventions concern both the 
ways that you must pass and return arguments to and from
a function and also who is responsible for 
preserving the contents of any registers that are used.
The system used to allow for arguments to be 
passed and registers to be saved is called the stack. The
code you write in this second part of the assignment 
must correctly use the stack and obey the
conventions to successfully interact with the C 
code that makes up the rest of the program. On the
WRAMP architecture the "top of stack" is pointed 
to by the register \reg{sp} and the stack grows downwards.
The WRAMP calling conventions are detailed fully in a
lecture handout, but in summary:

\begin{itemize}
\item All arguments are passed on the top 
of the stack to the callee function. The stack pointer will
point to the first argument.
\item Values returned by a function are returned in register one (\reg{1}).
\item All registers used by the callee function must be preserved and 
returned in tact to the caller
function. This means that if you use any register you must 
first save it before you use it, and
restore it before you return. The only 
exception to this rule is \reg{1}. This register is allowed to
be overwritten by any function, whether the 
function returns a value or not.
\item Any function that wishes to call another function 
must save their return address from \reg{ra}
onto the stack. The jal command used to call 
the next function will otherwise overwrite your
return address. Leaf functions need not do this.
\end{itemize}

\subsection{Question 2}
\emph{
 You are to write your own version of the 
\src{counter\_ctrl} function in WRAMP assembly. This
function will be called with a single argument, that 
being the value on the switches. Your code
must call the function counter that you wrote in \emph{Q1} 
of this assignment. You should pass the
value on the switches directly to the \src{value} 
argument of the \src{counter} function. In your
\src{counter\_ctrl} function the \src{direction} argument 
passed to \src{counter} should be set to \src{1}
if the upper 4 bits of the switches (bits 4 to 7) are 
a greater than or equal to the lower 4 bits
(bits 0 to 3) of the switches, otherwise direction should 
be set to \src{0}.
For example, if the switches are set to \src{0x92} 
(\src{1001 0010}) you should call the counter
function with \src{value} set to \src{0x92} and 
\src{direction} set to \src{1}.
If the switches are set to \src{0x26} (\src{0010 0110}) you 
should call the counter function with \src{value}
set to \src{0x26} and \src{direction} set to \src{0}.
}

You should call the file you save your \src{counter\_ctrl} 
function in \filename{my\_ctrl.s}. You must declare
your \src{counter\_ctrl} function global by placing the 
following at the top of your code:

\begin{verbatim}
     .global counter_ctrl
\end{verbatim}

To assemble your new function you should sue the 
following wasm command:

\begin{verbatim}
     wasm my_ctrl.s
\end{verbatim}

To link your counter and counter\_ctrl functions together 
along with the provided main and
library functions you should use the following 
\program{wlink} command on one line:

\begin{verbatim}
     wlink -�o ex3_2.srec counter.o my_ctrl.o
           /home/201/ex3/main.o /home/201/lib.o
\end{verbatim}

You should now have a file named \filename{ex3\_2.srec} 
that you can upload and test on the REX board.
During verification the demonstrator may require 
that you can prove that your \src{counter\_ctrl}
function does not modify any registers without saving 
and restoring them before returning to the caller.

\section{Appendix - Library Routines}

\begin{description}
\item[void putch(char character)] The subroutine outputs a 
single ASCII character to the terminal. The character to be output is
passed as the first argument.
%
\item[void putstr(char *str)] The subroutine outputs
a null terminated string to the terminal. The address in memory of the
first character in the spring is passed as the first 
argument. Uses printf so any percentage signs
may cause unexpected side-effects.
%
\item[int readnum(void)] Subroutine to read a decimal 
integer number from the keyboard and convert it to a 32 bit 2's
compliment number. Number is echoed to terminal. The 
subroutine exits when a non-digit
character is entered and returns the number read in \reg{1}.
%
\item[void writenum(int num)] Subroutine to write a 32
 bit 2's compliment number to the terminal as a decimal integer
number. The number to output is passed as the first argument.
%
\item[int readswitches(void)]
When called this subroutine returns, in \reg{1}, the current 
value on the switches in the low order
bits (0 to 7). Bits 8 through 31 of \reg{1} are cleared 
(set by zero) by this subroutine.
%
\item[void writessd(int num)] When called this subroutine 
will display the lowest order byte (bits 0 to 7) of the first argument as
a pair of hexadecimal digits to the seven segment displays. 
For example if num contains
0x0000005a when the subroutine is called then "5a" will 
be displayed on the seven-segment
displays.
%
\item[void exit(void)] This is a non-returning 
subroutine that when called returns control back to the monitor.
%
\item[void delay(void)] This subroutine will return 
approximately 0.25 seconds after it is called.
\end{description}

\thispagestyle{rcsfooters}
\pagestyle{rcsfooters}
\end{document}
