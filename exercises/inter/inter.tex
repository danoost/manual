\documentclass[a4paper,10pt]{article}
\setlength{\parindent}{0ex}
\setlength{\parskip}{1.5ex}

\usepackage[dvips]{graphicx}
\usepackage{epsfig}
\usepackage[a4paper, hmargin=25mm, vmargin=30mm, nohead]{geometry}

\usepackage{fancyhdr}
\fancypagestyle{rcsfooters}{
\fancyfoot[L]{\small $ $RCSfile: inter.tex,v $ $}
\fancyfoot[R]{\small $ $Revision: 1.3 $ $}
}

%INCLUDE OUR GLOBALS
\usepackage{rex201}
\usepackage{styles201}
\usepackage{ex201}

\renewcommand{\headrulewidth}{0pt}

\begin{document}

\EXHEADING{\INTERNO}{\INTERTITLE}{%
Verification Date:~\INTERDUE\\
Exercise Test Date:~\TESTTWODATE\\  	
}


\section{Assessment}

To show that you have completed this exercise to a satisfactory
standard you will be required to demonstrate at a verification session
that the program you wrote in \textbf{Question~\ref{ques:final}} of
this exercise runs correctly. In addition the marker may ask one or
two questions so they establish that you understand what you have
done.  These verification sessions will be held in \ASSESSROOM during the
following two times:

\begin{itemize}
\item \INTERDUE~\MORNINGASSESS
\item \INTERDUE~\AFTERNOONASSESS
\end{itemize}

In an attempt to ensure that the verification process runs smoothly
six machines in \ASSESSROOM will be booked for this purpose during the two
sessions. When it is your turn to have your program verified you
should log onto one of these machines and get your program ready to
run on the REX board. A marker will then verify your program. Once
your program has been verified you should log off immediately to allow
the next person to prepare to have their program verified.

In addition to the verification process there will be a closed book
exercise test on~\textbf{\TESTTWO}. The test can cover any material
from this exercise.

\section{Objective}

The purpose of this exercise is to reinforce the ideas presented in
lectures on interrupts and exceptions. To achieve this goal you will
be asked to write WRAMP code that utilises the interrupt and exception
mechanisms provided in the WRAMP processor.

\section{Introduction}

In this exercise you will make use of interrupts to write a program
that carries out two activities at the same time.

The final goal for your program is to read the switches and display
the value of the switches on the seven segment display while
displaying on the terminal the time since the code began to run (in
minutes and seconds).  In working toward this goal you will complete
a number of simpler stages.

Most students find this exercise challenging.  It is probably the most
challenging in the course.  {\em Please} be kind to yourself: start
early; make sure you understand what you're doing before you begin, if
not ask at a tutorial; work in small steps; save your last working
version before changing it.

This exercise assumes that you have read and understand the sections
of your manual on I/O Devices and Exceptions. These sections
provide, amongst other things, a step by step approach to writing your
exception handler and configuring the timer.

As this exercise is a first introduction to exceptions we do not
expect that everyone will be able to get a fully compliant exception
handler as discussed in the exception guide. Instead the aim of this
exercise is to get to a point where you have code that achieves all of
the objectives but does not preserve the contents of any registers
that it uses. This means that different registers must be used in the
exception handler to the primary code.

The way to write a fully compliant exception routine that carefully
preserves the contents of all registers will be covered in the
upcoming lectures. As such the next exercise dealing with exceptions
will build from the code created in this exercise and add the required
functionality to preserve all registers.

Students are welcome to attempt to obtain a fully compliant exception
routine in this exercise as an extension. All the technical details of
how to do this are discussed in fully in the manual.

\section{Questions}


\begin{enumerate}
\item

Take a copy of the code you wrote for question 1 of exercise 5. This
program should continually read the value of the switches and output
this value to the seven segment displays. Following the steps outlined
in the exception guide add code to this program to handle the user
interrupt button. When you press the interrupt button the handler should print
a single \src{`X'} to the serial port connected to the Linux machine.

\item 

Change from using the user interrupt button to cause interrupts to
using the timer. Set the timer so that it generates an interrupt every
second.  Leave the exception handler the same so that it prints a
single character to the serial port. In the mainline continue to read
the switches and write the value to the seven segment display. All the
details on the way to configure and use the timer are provided in the
I/O device guide.

\item 

Modify your exception handler so that it prints the time in seconds
since the program started.

\item 

Modify your exception handler so that it prints the time in minutes and seconds
since the program started.

\label{ques:final}
\end{enumerate}

\section{Extensions}

These are offered as extensions that you may wish to try if you
complete the base requirements early.

\begin{enumerate}

\item

Add to your time printing the ability to display the time in more
units. For example you could add the ability to display hours, days
and years. To test this you may wish to change your exception routine
to update the time faster than once per second.

\item

Make your exception routine compliant. This is discussed in full in
the exception guide, but basically the aim is for the exception
handler to preserve the contents of all the registers it uses during
the handler.

\end{enumerate}

\thispagestyle{rcsfooters}
\pagestyle{rcsfooters}
\end{document}

