\section{Introducing the Basys3 and WRAMP}

This manual will guide you through the use of Basys3 FPGAs that are programmed
with the WRAMP architecture. Only the Basys3 itself is required for most of the
functionality, but the second serial port requires a Micro-USB PMOD attachment.
WRAMP can be used without this attached, but the second serial port will not function.

WRAMP is designed to be a simple architecture which can help you gain a practical
understanding of the internal workings of a computer. The initial implementation
on REX boards was developed at the University of Waikato by Dean Armstrong, as
part of an undergraduate degree. The implementation for Basys3 FPGAs was developed
by Daniel Oosterwijk and Tyler Marriner at the University of Waikato.

\subsection{The WRAMP CPU}
The \BI{Central Processing Unit} (\BI{CPU}) is responsible for carrying out 
instructions.
CPUs have developed at an astounding rate.  From the humble Z80 to the 386, and
beyond the GHz barrier.  But as the speed increases, so does
the complexity.  Due to this complexity it is difficult learning to program
at a low level. 

The WRAMP CPU was designed to solve this problem. It is much easier to 
understand because it does not have the quirks of performance orientated CPUs.
Some features of the WRAMP CPU are:
\begin{itemize}
\item An easy to understand instruction set.  All instructions conform
to a clear and consistent structure.
\item All operations are carried out on 32 bit words.
\item Only a single instruction is executed at a time.
\item Memory addressing is easy to understand.
\end{itemize}

\subsection{The REX board}
For a CPU to be usable a motherboard is needed.  The REX board was developed
to host the WRAMP CPU. The REX boards can be found mounted on the walls of 
Lab 1.  Each is connected to the Linux machine below it and the terminal 
above it.

The REX board is made up of the following components:

\begin{description}
\item[CPU] The WRAMP processor is the brain of the REX board.
%
\item[RAM] Capable of storing 128K words, the RAM resides at addresses 0x00000
to 0x1FFFF.  The last 2K words are reserved for use by the monitor program.  The rest 
of the RAM is used to store and execute your programs, which, by default, are 
loaded starting from address 0x00000.
%
\item[ROM] Capable of storing 256K words, the ROM resides at addresses 
0x80000 to 0xBFFFF.  It is used to store the \WRAMPmon\ monitor program.
%
\item[Serial Interface] Allows the WRAMP processor to interface with two serial
ports.  One is connected the Linux machine next to it.  The other connects to
the terminal on the shelf above.  The terminal will be used in later 
exercises.
%
\item[Parallel Interface] Allows the WRAMP processor to interface with: 
\begin{itemize}
\item a set of eight switches from which the CPU can read
\item two pushbuttons from which the CPU can read
\item two seven segment displays to which the CPU can write
\end{itemize}
\item[Timer] Counts down from a certain value and can interrupt the CPU when 
finished.
\item[Control Panel] Provides the user with the ability to control execution
of the WRAMP CPU.  It also has an LCD display showing the current status
of the CPU.
%
\end{description}
Use of these I/O devices will be covered in chapter \ref{chapter:io}.


\subsection{WRAMPmon}
The REX board has installed on it monitor software known as \WRAMPmon.
It communicates with the user through the Linux machine connected to
the serial port. \WRAMPmon\ is used to upload your programs to the REX
board and to debug them.  This will be covered in more detail in
section \ref{intro:wrampmon}.
%
%
%  SECTION - Assembly
%
\section{Introduction to the WRAMP Assembly Language} %???
%
To help you get started we will begin by analysing,  assembling, 
executing and debugging a simple WRAMP assembly language program.
Using \program{emacs} (or your preferred text editor) enter the  
program listed in Figure \ref{simple_prog}. Save it as \filename{intro.s}. A digital copy is also available at \filename{/home/comp200/manual/intro.s}; you can simply copy this to avoid retyping the program.

%%%
%
% Example Code
%
%%%
\begin{figure}[btp]
\begin{center}
%\begin{footnotesize}
\begin{tabular}{|p{15cm}|}
\hline
\begin{verbatim}
 .text
 .global main
 main:
      # Get the address of the welcome message
      la     $2, welcome_msg
      # Display the message
      jal    putstr
      # Clear our sum register
      add    $4, $0, $0
      # Initialise the loop counter
      addi   $5, $0, 1
 loop:
      # Read a number from the user
      jal    readnum
      # Add it to our running total
      add    $4, $4, $1
      # Increment our loop counter
      addi   $5, $5, 1
      # Test to see if we have done all 4 numbers
      slti   $1, $5, 4
      # Keep looping until 4 numbers have been entered
      bnez   $1, loop
      # Get the output message
      la     $2, output_msg
      # Print it
      jal    putstr
      # Move our sum into register $2 to display it
      add    $2, $0, $4
      # Print out the number
      jal    writenum
      # Return to the monitor
      j      exit
 .data
      # This is our welcome message
 welcome_msg:
      .asciiz  "Welcome to the world of WRAMP!!!\n\nPlease
               type four numbers, pressing enter after each:\n"
      # This is the output message
 output_msg:
      .asciiz    "The sum of the numbers is : "
\end{verbatim}%$

\\
\hline
\end{tabular}
%\end{footnotesize}
\end{center}
\caption{WRAMP Assembly Language Program (\filename{/home/comp200/manual/intro.s})}
\label{simple_prog}
\end{figure}
%%%%
%%%
%%
%

This program first displays a message asking you to enter four numbers.  It
is then meant to input four numbers, add them up and display the result. 
If you have noticed the bug in it, don't worry as we will correct this later.

As you read through the source file look for the following:

\begin{description}
\item[comments] are lines beginning with a hash (\src{\#}).  These work in 
the same way as \src{//} comments in C and are ignored by the assembler.
%
\item[labels] such as \src{main:} are used to refer to a location in the
memory.  This may be used to move execution to a different area of code,
or to refer to data in the memory.
%
\item[directives] are commands beginning with a period (\src{.}).  They
pass information to the assembler.
%
\item[subroutine calls] are used in this program to perform hardware
operations such as reading from the serial port.  For example the
lines \src{jal putstr} and \src{jal readnum} are subroutine calls.
%
\item[WRAMP instructions] such as \src{add \reg{4}, \reg{0}, \reg{0}}  get 
converted to machine code by the assembler.  They can then be executed by the 
WRAMP CPU.
%
\end{description}

\subsection{Labels}
A label is a means of referencing a location in memory.  Rather than
specifying absolute addresses when accessing data in memory, a label
can be used.  This makes it much easier for the programmer to write
and understand the code.  Labels can also be declared within the
code. They can then be used as a destination for branch and jump
instructions.

A label declaration consists of a name (containing no spaces) followed
by a colon (\src{:}).

The example code in figure \ref{simple_prog} has labels \src{main},
\src{loop}, \src{welcome\_msg}, \src{output\_msg}.  The labels
\src{main} and \src{loop} refer to locations within the code.  The
labels \src{welcome\_msg} and \src{output\_msg} refer to data.





\subsection{Directives}
Directives (commands preceeded by a \src{.}) do not become part of the
final executable.  They are used to pass information to the assembler,
like \src{\#} directives do in C.

\subsubsection{\text, \data and \bss}
A program written in WRAMP assembly can be split into three sections.
These are used to seperate executable code, initialised data, and
uninitialised data.  The directives \text, \data and \bss are used to
do this.  When the assembler encounters one of these directives it
knows that all the code following it belongs in the given section.

The \text section contains WRAMP assembly instructions which will be
converted to machine language which can be executed on the WRAMP CPU.

%The \data section is used for initialised data.  
Within the \data section, space in the memory can be reserved and
initialised.  This is useful for strings, constants, and variables
which have an initial value.

The \bss section allows memory space to be reserved but not initialised.  The
advantage of this is that space can be reserved in chunks using a single
command.  This feature makes the \bss section useful for arrays. 

\subsubsection{Assigning Space}
Within the \data and \bss sections memory space is assigned using directives. 
To be able
to access this you need to place a label before the assigning directive. In the
example code, the label \src{welcome\_msg:} refers to the memory reserved 
by the \asciiz directive.
%
\begin{verbatim}
   welcome_msg:
      .asciiz "Welcome to the world of WRAMP!!!\n\n"
\end{verbatim}



Some of the memory assigning directives you may come across:
\begin{description}

\item[\word \src{n}]  This assigns one word of memory space and 
initialises it to the number \src{n}.

\item[\asciiz \src{"str"}] reserves and initialises space 
for a NULL terminated ASCII string (\src{"str"}).
This means that the string is followed immediatly by a NULL
character (\src{0}).  The NULL character can then be used
to identify the end of the string.

\item[\ascii \src{"str"}] reserves and initialises space for 
the ASCII string \src{"str"} without NULL terminating.

\item[\Space \src{n}] is used to allocate a chunk of space 
of size \src{n} in the \bss section.

\end{description}
The \ascii and \asciiz directives can not be used in the \bss section
as they initialise the space they reserve.  The \word directive may be
used as long as it is not provided with an argument.

\subsubsection{The \Global Directive}
When linking multiple files, we need to share functions and data.
However we do not want to expose all the labels in a file.
Only labels declared as global (using the \Global directive) are
accessible outside the current file.  

In the example program we have declared the `entry point' \src{main} as 
global. To do this we used the following directive:
\begin{verbatim}
.global main
\end{verbatim}


\subsection{Registers}
Registers are the CPU's equivalent of variables.  There are 16 general
purpose registers numbered 0 to 15. They can be used for temporarily
storing data while it is used in operations.  Registers are generally
referred to by a \$-sign followed by a number.  For example \reg{0}
refers to register zero. The contents of registers can be transferred
to and from main memory. This done using load and store instructions
which are covered in a later section.


Some of the general purpose registers have special uses:

Register \reg{0} is always zero.  Any attempts to write to it are
ignored.  This provides a constant source of zero that can be used for
comparing and initialising registers.

Register fourteen is denoted \reg{sp}. This register is defined by
convention to be the stack pointer.  While the hardware imposes no
special conditions on this register, failure to follow this convention
may affect the ability of code to interoperate with other software.

Register fifteen is denoted \reg{ra}. It is defined by convention to
be the subroutine return address register.  When a jump and link
instruction is executed this register is loaded with the address of
the next instruction after the jump and link. We will discuss stacks
and this register in a later chapter.

In addition to these general purpose registers is a set of special
purpose registers.  The use of these is covered in chapter
\ref{chapter:exceptions}.

Various types of instructions are introduced below.

\subsection{Instructions}

Executable machine code is create by assembling WRAMP instructions.
Instructions tell the WRAMP CPU what to do.  For example an \src{add} 
instruction will add the contents of two registers and place the result 
in another.

\subsubsection{Arithmetic}
Arithmetic instructions come in four forms, which are listed 
below, using \src{add} as an example:


\src{add \regd, \regs, \regt} \\*
  Simply performs the specified operation,
  on \regs\ and \regt, placing the result in \regd.  For example 
  \mbox{\src{add \$1, \$2, \$0}} will add the contents of registers
  \reg{2} and \reg{0} then place the result in \reg{1}.
  \\

\src{addi \regd, \regs, Immediate} \\* 
  This is known as immediate form.
  The argument \src{Immediate} is a constant.  Thus the specified
  operation is performed on \regs\ and \src{Immediate}, and the result
  placed in \regd.  For example \mbox{\src{addi \$1, \$2, 4}} will add
  \src{4}\ to the contents of \reg{2}\ and place the result in
  \reg{1}.  
  \\

\src{addu \regd, \regs, \regd} \\* 
  The \emph{u}\ implies that this is
  an unsigned instruction.  Thus it is assumed that both the operands
  are positive.  This is important when we consider twos complement
  representation of negative numbers.  For example \\ \mbox{\src{1111
  1111 1111 1111 1111 1111 1111 1111}} represents \src{-1} when
  treated as signed, and 0xFFFFFFFF, a very large positive number,
  when treated as unsigned.  \\

\src{addui \regd, \regs, Immediate} \\*
  This is the combination of the two forms above.  It performs an unsigned
  operation on \regs\ and the \src{Immediate} value.
  \\

For a full listing of arithmetic instructions see  Appendix 
\ref{appendix:instr}.


\subsubsection{Memory I/O}
The WRAMP CPU has a 20 bit memory space.  Of this the memory locations 
0x00000 to 0x1FFFF are RAM.

Unlike other CPUs the WRAMP has only a single method of referencing
external memory.  It consists of a base address added to an offset.
The base address is specified as a constant in the instruction.  The
offset is the contents of a specified register.

The notation for this is \src{base(\regs)}.  The \src{base} can either
be a label or an integer.  To load from the memory location specified
by a label, \reg{0} can be used for \regs.  For the first few chapters
we will use this method of referencing the memory. Figure \ref{memio}
shows an example. It loads a word of data from memory into a register,
adds 1 to it, then stores it back into the memory.


\src{lw \regd, base(\regs)}\\*
Used to get the contents of the specified memory location and place it 
in the register \regd.

\src{sw \regd, base(\regs)}\\*
Will place the contents of \regd\ into the specified memory location.
\\


\begin{figure}[btp]
\begin{center}
%\begin{footnotesize}
\begin{tabular}{|p{15cm}|}
\hline
\begin{verbatim}
.text
     . . .

     #Read from the memory location `counter' into $4
     lw $4, counter($0)

     #Add 1 to $4
     addi $4, $4, 1

     #Store the contents of $4 into the memory location `counter'
     sw $4, counter($0)

     . . .
     
.data
     # This is our counter
counter:
     .word 0 
\end{verbatim}%$
\\
\hline
\end{tabular}
%\end{footnotesize}
\end{center}
\caption{Memory I/O Example}
\label{memio}
\end{figure}

Sometimes you will need to know the address of a variable.  The 
instruction `load address' ( \src{la} ) is used for this. It has the 
structure:
\\*

\src{la \regd, label} \\*
Load the address that \src{label} refers to into register \regd.  

\subsubsection{Set Instructions}
These instructions are used to compare either two registers, or a
register and an immediate value.  They take the same form as
arithmetic instructions.  If the test passes the destination register
is set to \src{1}, otherwise it is cleared to \src{0}.

For example to test if registers \reg{2} and \reg{3} were equal and store the
result in \reg{1} we would use the `set if equal' instruction: 
\src{seq \$1, \$2, \$3}.

A full listing of test instructions can be found in the instruction set reference, Appendix \ref{appendix:instr}.

\subsubsection{Program Control}
We move execution to a different part of the program using two types of 
instruction: the unconditional jump, and the conditional branch.  

\textbf{Jump Instructions}\\*
The jump instructions simply move execution to a different line of
code.  They are unconditional and so do not perform any tests first.
There are four different varieties:

\src{j  label}  - Jump\\*
Jumps to the specified label.

\src{jal label} - Jump and link\\*
Stores the address of the next instruction into the return address register \reg{ra}. It then jumps to the specied label. This is used when calling subroutines.

\src{jr \regs}  -  Jump Register\\*
Jumps to the location specified by the contents of the register \regs.  
Because the WRAMP CPU has only a 20 bit address space the upper 12 bits 
of the register are ignored.  Jump register is typically used to 
return from a subroutine using \src{jr \reg{ra}}

\src{jalr \regs} - Jump and Link Register\\* A combination of
\src{jal} and \src{jr}. It works in the same way as \src{jr} but it
stores the address of the next instruction into the return address
register before jumping. 

\textbf{Branch Instructions}\\*
Branching instructions are conditional, they look at whether a
register is zero or not, and act on that information.  There are two
varieties: `branch if equal to zero' \src{beqz} and `branch if not
equal to zero' \src{bnez}.  Both of these instructions take a single
register and a label as arguments.

For example the `branch if not equal to zero' command 
`\src{bnez \$1, loop}' will branch to label 
\src{loop} if \reg{1} is non-zero.

\begin{comment}
\subsection{I/O Routines}

In the example program you will see we have used the routines
\src{putstr}, \src{readnum} and \src{writenum} to perfrom input
and output.  These routines are in the file \LIBEXTWO, which we will 
\BI{link} to.  Here is an overview of the routines:
\begin{description}
\item[\src{putstr}] is used to write a string to the Linux machine.  Put the
address of the string into register \reg{2}.
\item[\src{readnum}] reads a number from the Linux machine, leaving it in
register \reg{1}.
\item[\src{writenum}] will write the number in register \reg{2} to the 
Linux machine.
\end{description}
\end{comment}

\subsection{Assembling and Linking}
Before we can execute our program on the REX board we need to translate it 
into machine code, so that the WRAMP CPU can understand it.  This is a two
step process.

\subsubsection{Assembly}
We firstly have to \BI{assemble} the source code using a program
called an \filename{assembler}.  This involves checking that all the
directives and instructions within the program make sense (are
syntactically correct).  The program is then translated into a file
called an \BI{object file}. This contains portions of machine code
along with other information about the program.

To assemble our file we use the WRAMP \BI{assembler}, \program{wasm}:
\begin{verbatim}
    wasm intro.s
\end{verbatim}

There should now exist a file called \filename{intro.o}.  This is the 
\BI{object file}.

\subsubsection{Linking}
Next we must \BI{link} the \BI{object file}.  A program may be
comprised of many separate parts.  Even our example program contains
functions (subroutines) contained in a separate \BI{library}. Linking
involves joining all these parts together to create a final program.

The WRAMP \BI{linker} is called \program{wlink}.  We will use it to
link our object file to the library object file in the COMP200 home directory.
This file \BI{does not} need to be copied into your home directory.
Issue the command:
\begin{verbatim}
     wlink -o intro.srec intro.o /home/comp200/manual/lib_manual.o
\end{verbatim}

This command links together the object files to create file called
\filename{intro.srec}.  This file is in a form known as S-Record, which
is suitable to be uploaded to the REX board and executed.

%
%
% SECTION - WRAMPmon
%
\section{Introduction to WRAMPmon}
\label{intro:wrampmon}

The \WRAMPmon\ monitor is a program which runs on the REX boards.  It
provides you with basic facilities for interacting with the REX board.
It is important for hardware and software development, debugging,
testing and troubleshooting.  It uses a command line interface to
perform these functions.

\WRAMPmon\ communicates through the serial port that is connected to the Linux 
machine.  By running a terminal program on the Linux machine we can talk
to the REX board.


\subsection{WRAMPmon commands}
A set of commands is provided for the user to interact with \WRAMPmon\
including:

\begin{itemize}
\item a help command (\src{help}, or \src{?})
\item commands to view and alter the memory and register contents
\item a command to upload your programs
\item commands to execute programs in the memory
\item commands for debugging
\end{itemize}

For a listing of these commands see Appendix \ref{appendix:wrampmon}.

\subsection{Getting started}
The REX board uses serial ports as its primary form of communication.
The main serial port on the REX board is connected to a Linux
machine. The Linux machine runs a terminal program, called
\filename{remote}, which transmits anything typed on the keyboard
directly to the REX board, and displays any text received from REX.

To execute the terminal program type \verb|remote| from a console.
Once you have \verb|remote| running you will need to reset the REX
board by pressing the red \src{RESET} button.  You should see
something like Figure \ref{wrampmon}.

\begin{figure}[hbp]
\begin{center}
%\begin{footnotesize}
\begin{tabular}{|p{15cm}|}
\hline
\begin{verbatim}
      +------------------------------------------------+
      |                 WRAMPmon 0.6                   |
      | Copyright 2002, 2003 The University of Waikato |
      |                                                |
      |          Written by Dean Armstrong             |
      +------------------------------------------------+
      
      Type ? and press enter for available commands.
      
       > 
\end{verbatim}
\\
\hline
\end{tabular}
\end{center}

\caption{\WRAMPmon}
\label{wrampmon}
\end{figure}

Type \src{?} to obtain a list of commands available within \WRAMPmon.

Help on individual commands can be obtained by typing the command and then a 
question mark, eg:
\begin{verbatim}
     load ?
\end{verbatim}

Each command gives the required format and available options.
An item enclosed in square brackets, \src{[}~and~\src{]},
indicates an optional item. An item enclosed in angled brackets, 
\src{<}~and~\src{>}, indicates a compulsory item.

If you have any problems with the monitor not responding at any stage you 
should always be able to get back to
the board's initial state with the start-up message displayed by pressing the 
red \src{RESET} button on the REX board.

\subsection{Uploading and Executing your Program}
Before you can run your program on the REX board it must first be loaded into 
the board's memory. The easiest
way to do this is to set the board into load mode and use the upload option 
within the \program{remote} program to send
the executable file from the Linux machine to the board. To place the board 
into load mode, type the following
command (then press enter) at the \WRAMPmon\ prompt:
\begin{verbatim}
     load
\end{verbatim}

After hitting return there should be a reminder response from the board indicating 
how to send an executable file. To send the executable type:

\verb|     <ctrl>-a| and then \verb|s|

Where \verb|<ctrl>-a| means press and hold the \src{control} key and hit 
the \src{a} key and then release both keys.
The \program{remote} program does not pass this key sequence on to the REX 
board but instead enters a command
mode. When you press \src{s}, remote will switch into upload mode. A
dialog box appears which asks you to enter the name of the file you wish to 
upload. Type in the following file name:
\begin{verbatim}
     intro.srec
\end{verbatim}

After the file name has been entered a series of dots should appear on the 
screen indicating that the file is being
uploaded, followed by a message which tells you that is has completed. You 
should then press the enter key to
leave the upload mode of \src{remote} at which stage the \WRAMPmon\ prompt 
should reappear.

Now that the program has been uploaded, you can run it by typing the command:
\begin{verbatim}
     go
\end{verbatim}

This will start executing your program from the entry point defined by the 
\src{main} label. The program will
prompt you to enter four numbers. You should continue to type a number and 
hit enter until the program
outputs the resulting sum. The output of your program should appear on your 
screen. When the \WRAMPmon\
prompt appears again it indicates that your program has finished executing.


\subsection{Debugging your Program}
The program that you have just run on the REX board should have read four 
numbers from the keyboard and
output their sum to the terminal. However, when run, only three numbers
are read before their sum is output.

In this section of the exercise you are going to use the debugging 
features of \WRAMPmon\ to locate the bug that
causes this problem.

The debugging commands within \WRAMPmon\ allow you to trace the execution 
of a program (i.e. follow the path
of execution) by inserting breakpoints.

If a breakpoint is set at a certain instruction, then when the program
is running, and that instruction is about to be executed, control will
be returned to the monitor. From here, you can view register and
memory contents, and maybe resume execution, or step through the
program one instruction at a time to try and identify bugs.  The
commands for setting, removing and viewing breakpoints are \src{sb},
\src{rb}, and \src{vb} respectively. The help facilty in \WRAMPmon\
fully describes the operation of these commands. For example to find
out more about the set breakpoint command type
\begin{verbatim}
     sb ?
\end{verbatim}

Insert a breakpoint at the beginning of the loop in the above program 
(ie. at memory location \src{0x00004}), and
execute the program using the \src{go} command.
When a break does occur, use the view registers command to find out the 
contents of the registers. You should also notice that the breakpoint 
register dump contains identical information
to the \src{vr} command.

To continue execution type:
\begin{verbatim}
     cont
\end{verbatim}
This will cause the program to continue execution until either the
breakpoint is encountered again or the program completes
execution. Remember that the program will now wait for you to enter a
number before it will break again.

Another feature of the monitor that can be used to help debug you programs 
is the single-step command \src{s}.
This command causes only the immediately next instruction 
to be executed and then returns back to
\WRAMPmon.

To demonstrate the use of the single step command you are to reload
the program.  Create a new breakpoint at memory location
\src{0x00005}. Once this is done run the program using the \src{go}
command.

Once you have encountered the breakpoint try single stepping through
the program using the \verb|s| command. Pay careful attention to the
loop counter in \reg{5} and the instructions that increment and test
it. As we are only interested in the loop counter at this stage we
will skip the \src{readnum} subroutine.  We do this using the `step
over' \src{so} command.  When you reach the instruction \src{jal
readnum} instead of typing `\src{s}' type `\src{so}'.

At some point you may wish to remove the breakpoint using the
\src{rb} command and then use \src{cont} to make the program run to
completion.

As you may have realised the program is designed with a bug.  Although
it is intended to add 4 numbers it will only add 3.  You should now use
the debugging methods we have just covered to determine the
bug.  Then correct it and assemble, link and load the program to confirm
it operates correctly.
