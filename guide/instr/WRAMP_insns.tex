\documentclass[12pt]{report}
\usepackage{a4}
\usepackage[dvips]{graphicx}
\usepackage{verbatim}

%INCLUDE OUR GLOBALS
\usepackage{styles201}
\usepackage{rex201}


\begin{document}

\setcounter{secnumdepth}{0}
%\pagenumbering{roman}

\section{WRAMP General Purpose Registers}

\begin{small}
The WRAMP general purpose register file consists of 16 registers, each
being 32 bits wide. The hardware imposes special uses on only two of
these.  Certain software register use conventions have been applied to
some of the remaining registers, but in essence they remain true
general purpose registers, as the hardware does not restrict their
use.
\end{small}

\vspace{-4ex}
\begin{figure}[h]
\begin{center}
\begin{tabular}{|l|l|}
\multicolumn{1}{@{}p{4ex}}{}
& \multicolumn{1}{@{}p{50ex}}{}\\ 
\hline
Register &Description\\
\hline
\texttt{\$0} & Hardwired zero\\
\texttt{\$1} - \texttt{\$13} & General purpose registers\\
\texttt{\$sp} & Stack pointer\\
\texttt{\$ra} & Return address register\\
\hline
\end{tabular}
\end{center}
\caption{The WRAMP General Purpose Registers}
\label{wramp_regs}
\end{figure}

\begin{small}
Register zero (denoted \texttt{\$0}) always contains the value
zero. Any writes to this register have the value discarded. This
provides a constant source of zero that can be used for comparing and
initialising registers.

The fourteenth register is denoted \texttt{\$sp}. This register is
defined by the conventions to be the stack pointer. While the hardware
imposes no special conditions on this register, failure to follow this
convention may affect the ability of code to interoperate with other
software.

The fifteenth register is denoted \texttt{\$ra}. It is defined to be
the subroutine return address register.  When a jump and link
instruction is executed this register is loaded with the address of
the next instruction after the jump and link. A return from subroutine
is performed by executing a jump to register \texttt{\$ra},
ie. \texttt{jr \$ra}.
\end{small}

\section{WRAMP Instruction Set Architecture}
\begin{small}
This section contains the details of the WRAMP instruction set.  All
machine instructions are listed, with their encoding and a brief
description of their function.  The instructions are grouped into
arithmetic instructions, bitwise instructions, test instructions,
branch instructions, memory instructions, and special instructions.

Each CPU instruction is a word (32 bits) in length. An instruction is
encoded in one of the three formats shown in figure
\ref{insn_encoding_formats}.
\end{small}

\newpage
% This is the table with the three instruction encoding formats
\begin{figure}[h]
{\bf I-Type instruction}
\itypeinsnformat
{\bf R-Type instruction}
\rtypeinsnformat
{\bf J-Type instruction}
\jtypeinsnformat
\vspace{1ex}
\begin{tabbing}
xxxx\=field name xxxxxxxx\=description \kill
\>\texttt{OPCode}\>4 bit operation code\\
\>\texttt{\regd{}}\>4 bit destination register specifier\\
\>\texttt{\regs{}}\>4 bit source register specifier\\
\>\texttt{\regt{}}\>4 bit source register specifier\\
\>\texttt{Func}\>4 bit function specifier\\
\>\texttt{Immediate}\>16 bit immediate field\\
\>\texttt{Address / Offset}\>20 bit absolute or relative address field
\end{tabbing}
\caption{WRAMP Instruction encoding formats}
\label{insn_encoding_formats}
\end{figure}


\subsection{Arithmetic Instructions}
\noindent
{\bf Addition}\\
\noindent
\texttt{add \regd, \regs, \regt}
\rtypeinsn{0000}{\regd}{\regs}{0000}{\regt}
Put the sum of register \regs{} and register \regt{}
into register \regd{}. Generate an overflow exception on signed overflow.
\vspace{3ex}

\noindent
{\bf Addition, immediate}\\
\noindent
\texttt{addi \regd, \regs, Immediate}
\itypeinsn{0001}{\regd}{\regs}{0000}{Immediate}
Put the sum of register \regs{} and the sign-extended immediate
into register \regd{}. Generate an overflow exception on signed overflow.
\vspace{3ex}

\noindent
{\bf Addition, unsigned}\\
\noindent
\texttt{addu \regd, \regs, \regt}
\rtypeinsn{0000}{\regd}{\regs}{0001}{\regt}
Put the sum of register \regs{} and register \regt{}
into register \regd{}. Generate an overflow exception on unsigned overflow.
\vspace{3ex}
\newpage

\noindent
{\bf Addition, unsigned, immediate}\\
\noindent
\texttt{addui \regd, \regs, Immediate}
\itypeinsn{0001}{\regd}{\regs}{0001}{Immediate}
Put the sum of register \regs{} and the zero-extended immediate
into register \regd{}. Generate an overflow exception on unsigned overflow.
\vspace{3ex}

\noindent
{\bf Subtraction}\\
\noindent
\texttt{sub \regd, \regs, \regt}
\rtypeinsn{0000}{\regd}{\regs}{0010}{\regt}
Put the difference of register \regs{} and register \regt{}
into register \regd{}. Generate an overflow exception on signed overflow.
\vspace{3ex}

\noindent
{\bf Subtraction, immediate}\\
\noindent
\texttt{subi \regd, \regs, Immediate}
\itypeinsn{0001}{\regd}{\regs}{0010}{Immediate}
Put the difference of register \regs{} and the sign-extended immediate
into register \regd{}. Generate an overflow exception on signed overflow.
\vspace{3ex}

\noindent
{\bf Subtraction, unsigned}\\
\noindent
\texttt{subu \regd, \regs, \regt}
\rtypeinsn{0000}{\regd}{\regs}{0011}{\regt}
Put the difference of register \regs{} and register \regt{}
into register \regd{}. Generate an overflow exception on unsigned overflow.
\vspace{3ex}

\noindent
{\bf Subtraction, unsigned, immediate}\\
\noindent
\texttt{subui \regd, \regs, Immediate}
\itypeinsn{0001}{\regd}{\regs}{0011}{Immediate}
Put the difference of register \regs{} and the zero-extended immediate
into register \regd{}. Generate an overflow exception on unsigned overflow.
\vspace{3ex}

\noindent
{\bf Multiplication}\\
\noindent
\texttt{mult \regd, \regs, \regt}
\rtypeinsn{0000}{\regd}{\regs}{0100}{\regt}
Put the product of the signed multiplication of register \regs{} and register \regt{}
into register \regd{}. Generate an overflow exception on signed overflow.
\vspace{3ex}
\newpage

\noindent
{\bf Multiplication, immediate}\\
\noindent
\texttt{multi \regd, \regs, Immediate}
\itypeinsn{0001}{\regd}{\regs}{0100}{Immediate}
Put the product of the signed multiplication of register \regs{} and the sign-extended immediate
into register \regd{}. Generate an overflow exception on signed overflow.
\vspace{3ex}

\noindent
{\bf Multiplication, unsigned}\\
\noindent
\texttt{multu \regd, \regs, \regt}
\rtypeinsn{0000}{\regd}{\regs}{0101}{\regt}
Put the product of the unsigned multiplication of register \regs{} and register \regt{}
into register \regd{}. Generate an overflow exception on unsigned overflow.
\vspace{3ex}

\noindent
{\bf Multiplication, unsigned, immediate}\\
\noindent
\texttt{multui \regd, \regs, Immediate}
\itypeinsn{0001}{\regd}{\regs}{0101}{Immediate}
Put the product of the unsigned multiplication of register \regs{} and the zero-extended immediate
into register \regd{}. Generate an overflow exception on unsigned overflow.
\vspace{3ex}

\noindent
{\bf Division}\\
\noindent
\texttt{div \regd, \regs, \regt}
\rtypeinsn{0000}{\regd}{\regs}{0110}{\regt}
Put the result of the signed integer division of register \regs{} by register \regt{}
into register \regd{}. Generate a divide-by-zero exception if the contents of \regt{} is zero.
\vspace{3ex}

\noindent
{\bf Division, immediate}\\
\noindent
\texttt{divi \regd, \regs, Immediate}
\itypeinsn{0001}{\regd}{\regs}{0110}{Immediate}
Put the result of the signed integer division of register \regs{} by the sign-extended immediate
into register \regd{}. Generate a divide-by-zero exception if the immediate value is zero.
\vspace{3ex}

\noindent
{\bf Division, unsigned}\\
\noindent
\texttt{divu \regd, \regs, \regt}
\rtypeinsn{0000}{\regd}{\regs}{0111}{\regt}
Put the result of the unsigned division of register \regs{} by register \regt{}
into register \regd{}. Generate a divide-by-zero exception if the contents of \regt{} is zero.
\vspace{3ex}
\newpage

\noindent
{\bf Division, unsigned, immediate}\\
\noindent
\texttt{divui \regd, \regs, Immediate}
\itypeinsn{0001}{\regd}{\regs}{0111}{Immediate}
Put the result of the unsigned division of register \regs{} by the zero-extended immediate
into register \regd{}. Generate a divide-by-zero exception if the immediate value is zero. 
\vspace{3ex}

\noindent
{\bf Remainder}\\
\noindent
\texttt{rem \regd, \regs, \regt}
\rtypeinsn{0000}{\regd}{\regs}{1000}{\regt}
Put the remainder of the signed division of register \regs{} by register \regt{}
into register \regd{}. Generate a divide-by-zero exception if the contents of \regt{} is zero.
\vspace{3ex}

\noindent
{\bf Remainder, immediate}\\
\noindent
\texttt{remi \regd, \regs, Immediate}
\itypeinsn{0001}{\regd}{\regs}{1000}{Immediate}
Put the remainder of the signed division of register \regs{} by the sign-extended immediate
into register \regd{}. Generate a divide-by-zero exception if the immediate value is zero.
\vspace{3ex}

\noindent
{\bf Remainder, unsigned}\\
\noindent
\texttt{remu \regd, \regs, \regt}
\rtypeinsn{0000}{\regd}{\regs}{1001}{\regt}
Put the remainder of the unsigned division of register \regs{} by the register \regt{}
into register \regd{}. Generate a divide-by-zero exception if the contents of \regt{} is zero.
\vspace{3ex}

\noindent
{\bf Remainder, unsigned, immediate}\\
\noindent
\texttt{remui \regd, \regs, Immediate}
\itypeinsn{0001}{\regd}{\regs}{1001}{Immediate}
Put the remainder of the unsigned division of register \regs{} by the zero-extended immediate
into register \regd{}. Generate a divide-by-zero exception if the immediate value is zero.
\vspace{3ex}

\noindent
{\bf Load high immediate}\\
\noindent
\texttt{lhi \regd, Immediate}
\itypeinsn{0011}{\regd}{0000}{1110}{Immediate}
Put the 16 bit immediate into the upper 16 bits of register \regd{},
and set the lower 16 bits to zero.
\vspace{3ex}
\newpage

\noindent
{\bf Load address}\\
\noindent
\texttt{la \regd, Address}
\jtypeinsn{1100}{\regd}{0000}{Address}
Put the zero-extended 20 bit address into register \regd{}.
\vspace{3ex}

\subsection{Bitwise instructions}

\noindent
{\bf And}\\
\noindent
\texttt{and \regd, \regs, \regt}
\rtypeinsn{0000}{\regd}{\regs}{1011}{\regt}
Put the result of the logical AND of registers \regs{} and \regt{} into register \regd{}.
\vspace{3ex}

\noindent
{\bf And, immediate}\\
\noindent
\texttt{andi \regd, \regs, Immediate}
\itypeinsn{0001}{\regd}{\regs}{1011}{Immediate}
Put the result of the logical AND of register \regs{} and the zero-extended immediate into register \regd{}.
\vspace{3ex}

\noindent
{\bf Or}\\
\noindent
\texttt{or \regd, \regs, \regt}
\rtypeinsn{0000}{\regd}{\regs}{1101}{\regt}
Put the result of the logical OR of registers \regs{} and \regt{} into register \regd{}.
\vspace{3ex}

\noindent
{\bf Or, immediate}\\
\noindent
\texttt{ori \regd, \regs, Immediate}
\itypeinsn{0001}{\regd}{\regs}{1101}{Immediate}
Put the result of the logical OR of register \regs{} and the zero-extended immediate into register \regd{}.
\vspace{3ex}

\noindent
{\bf Xor}\\
\noindent
\texttt{xor \regd, \regs, \regt}
\rtypeinsn{0000}{\regd}{\regs}{1111}{\regt}
Put the result of the logical exclusive-OR of registers \regs{} and \regt{} into register \regd{}.
\vspace{3ex}
\newpage

\noindent
{\bf Xor, immediate}\\
\noindent
\texttt{xori \regd, \regs, Immediate}
\itypeinsn{0001}{\regd}{\regs}{1111}{Immediate}
Put the result of the logical exclusive-OR of register \regs{} and the zero-extended immediate into register \regd{}.
\vspace{3ex}

\noindent
{\bf Shift left logical}\\
\noindent
\texttt{sll \regd, \regs, \regt}
\rtypeinsn{0000}{\regd}{\regs}{1010}{\regt}
Shift the value in register \regs{} left by the unsigned value given by the
least significant 5 bits of register \regt{}, and put the result in register \regd{},
inserting zeros into the low order bits.
\vspace{3ex}

\noindent
{\bf Shift left logical, immediate}\\
\noindent
\texttt{slli \regd, \regs, Immediate}
\itypeinsn{0001}{\regd}{\regs}{1010}{Immediate}
Shift the value in register \regs{} left by the unsigned value given by the
least significant 5 bits of the immediate, and put the result in register \regd{},
inserting zeros into the low order bits.
\vspace{3ex}

\noindent
{\bf Shift right logical}\\
\noindent
\texttt{srl \regd, \regs, \regt}
\rtypeinsn{0000}{\regd}{\regs}{1100}{\regt}
Shift the value in register \regs{} right by the unsigned value given by the
least significant 5 bits of register \regt{}, and place the result in register \regd{},
inserting zeros into the high order bits.
\vspace{3ex}

\noindent
{\bf Shift right logical, immediate}\\
\noindent
\texttt{srli \regd, \regs, Immediate}
\itypeinsn{0001}{\regd}{\regs}{1100}{Immediate}
Shift the value in register \regs{} right by the unsigned value given by the
least significant 5 bits of the immediate, and place the result in register \regd{},
inserting zeros into the high order bits.
\vspace{3ex}

\noindent
{\bf Shift right arithmetic}\\
\noindent
\texttt{sra \regd, \regs, \regt}
\rtypeinsn{0000}{\regd}{\regs}{1110}{\regt}
Shift the value in register \regs{} right by the unsigned value given by the
least significant 5 bits of register \regt{}, and place the result in register \regd{},
sign-extending the high order bits.
\vspace{3ex}
\newpage

\noindent
{\bf Shift right arithmetic, immediate}\\
\noindent
\texttt{srai \regd, \regs, Immediate}
\itypeinsn{0001}{\regd}{\regs}{1110}{Immediate}
Shift the value in register \regs{} right by the unsigned value given by the
least significant 5 bits of the immediate, and place the result in register \regd{},
sign-extending the high order bits.
\vspace{3ex}

\subsection{Test instructions}

\noindent
{\bf Set on less than}\\
\noindent
\texttt{slt \regd, \regs, \regt}
\rtypeinsn{0010}{\regd}{\regs}{0000}{\regt}
Set register \regd{} to 1 if register \regs{} is
less than register \regt{} according to a signed comparison, and 0 otherwise.
\vspace{3ex}

\noindent
{\bf Set on less than immediate}\\
\noindent
\texttt{slti \regd, \regs, Immediate}
\itypeinsn{0011}{\regd}{\regs}{0000}{Immediate}
Set register \regd{} to 1 if register \regs{} is
less than the sign-extended immediate according to a signed comparison, and 0 otherwise.
\vspace{3ex}

\noindent
{\bf Set on less than, unsigned}\\
\noindent
\texttt{sltu \regd, \regs, \regt}
\rtypeinsn{0010}{\regd}{\regs}{0001}{\regt}
Set register \regd{} to 1 if register \regs{} is
less than register \regt{} according to an unsigned comparison, and 0 otherwise.
\vspace{3ex}

\noindent
{\bf Set on less than, unsigned, immediate}\\
\noindent
\texttt{sltui \regd, \regs, Immediate}
\itypeinsn{0011}{\regd}{\regs}{0001}{Immediate}
Set register \regd{} to 1 if register \regs{} is
less than the zero-extended immediate according to an unsigned comparison, and 0 otherwise.
\vspace{3ex}
\newpage

\noindent
{\bf Set on greater than}\\
\noindent
\texttt{sgt \regd, \regs, \regt}
\rtypeinsn{0010}{\regd}{\regs}{0010}{\regt}
Set register \regd{} to 1 if register \regs{} is
greater than register \regt{} according to a signed comparison, and 0 otherwise.
\vspace{3ex}

\noindent
{\bf Set on greater than, immediate}\\
\noindent
\texttt{sgti \regd, \regs, Immediate}
\itypeinsn{0011}{\regd}{\regs}{0010}{Immediate}
Set register \regd{} to 1 if register \regs{} is
greater than the sign-extended immediate according to a signed comparison, and 0 otherwise.
\vspace{3ex}

\noindent
{\bf Set on greater than, unsigned}\\
\noindent
\texttt{sgtu \regd, \regs, \regt}
\rtypeinsn{0010}{\regd}{\regs}{0011}{\regt}
Set register \regd{} to 1 if register \regs{} is
greater than register \regt{} according to an unsigned comparison, and 0 otherwise.
\vspace{3ex}

\noindent
{\bf Set on greater than, unsigned, immediate}\\
\noindent
\texttt{sgtui \regd, \regs, Immediate}
\itypeinsn{0011}{\regd}{\regs}{0011}{Immediate}
Set register \regd{} to 1 if register \regs{} is
greater than the zero-extended immediate according to an unsigned comparison, and 0 otherwise.
\vspace{3ex}

\noindent
{\bf Set on less than or equal to}\\
\noindent
\texttt{sle \regd, \regs, \regt}
\rtypeinsn{0010}{\regd}{\regs}{0100}{\regt}
Set register \regd{} to 1 if register \regs{} is
less than or equal to register \regt{} according to a signed comparison, and 0 otherwise.
\vspace{3ex}

\noindent
{\bf Set on less than or equal to, immediate}\\
\noindent
\texttt{slei \regd, \regs, Immediate}
\itypeinsn{0011}{\regd}{\regs}{0100}{Immediate}
Set register \regd{} to 1 if register \regs{} is
less than or equal to the sign-extended immediate according to a signed comparison, and 0 otherwise.
\vspace{3ex}
\newpage

\noindent
{\bf Set on less than or equal to, unsigned}\\
\noindent
\texttt{sleu \regd, \regs, \regt}
\rtypeinsn{0010}{\regd}{\regs}{0101}{\regt}
Set register \regd{} to 1 if register \regs{} is
less than or equal to register \regt{} according to an unsigned comparison, and 0 otherwise.
\vspace{3ex}

\noindent
{\bf Set on less than or equal to, unsigned, immediate}\\
\noindent
\texttt{sleui \regd, \regs, Immediate}
\itypeinsn{0011}{\regd}{\regs}{0101}{Immediate}
Set register \regd{} to 1 if register \regs{} is
less than or equal to the zero-extended immediate according to an unsigned comparison, and 0 otherwise.
\vspace{3ex}

\noindent
{\bf Set on greater than or equal to}\\
\noindent
\texttt{sge \regd, \regs, \regt}
\rtypeinsn{0010}{\regd}{\regs}{0110}{\regt}
Set register \regd{} to 1 if register \regs{} is
greater than or equal to register \regt{} according to a signed comparison, and 0 otherwise.
\vspace{3ex}

\noindent
{\bf Set on greater than or equal to immediate}\\
\noindent
\texttt{sgei \regd, \regs, Immediate}
\itypeinsn{0011}{\regd}{\regs}{0110}{Immediate}
Set register \regd{} to 1 if register \regs{} is
greater than or equal to the sign-extended immediate according to a signed comparison, and 0 otherwise.
\vspace{3ex}

\noindent
{\bf Set on greater than or equal to, unsigned}\\
\noindent
\texttt{sgeu \regd, \regs, \regt}
\rtypeinsn{0010}{\regd}{\regs}{0111}{\regt}
Set register \regd{} to 1 if register \regs{} is
greater than or equal to register \regt{} according to an unsigned comparison, and 0 otherwise.
\vspace{3ex}

\noindent
{\bf Set on greater than or equal to, unsigned, immediate}\\
\noindent
\texttt{sgeui \regd, \regs, Immediate}
\itypeinsn{0011}{\regd}{\regs}{0111}{Immediate}
Set register \regd{} to 1 if register \regs{} is
greater than or equal to the zero-extended immediate according to an unsigned comparison, and 0 otherwise.
\vspace{3ex}
\newpage

\noindent
{\bf Set on equal to}\\
\noindent
\texttt{seq \regd, \regs, \regt}
\rtypeinsn{0010}{\regd}{\regs}{1000}{\regt}
Set register \regd{} to 1 if register \regs{} is
equal to register \regt{} according to a signed comparison, and 0 otherwise.
\vspace{3ex}

\noindent
{\bf Set on equal to immediate}\\
\noindent
\texttt{seqi \regd, \regs, Immediate}
\itypeinsn{0011}{\regd}{\regs}{1000}{Immediate}
Set register \regd{} to 1 if register \regs{} is
equal to the sign-extended immediate according to a signed comparison, and 0 otherwise.
\vspace{3ex}

\noindent
{\bf Set on equal to, unsigned}\\
\noindent
\texttt{sequ \regd, \regs, \regt}
\rtypeinsn{0010}{\regd}{\regs}{1001}{\regt}
Set register \regd{} to 1 if register \regs{} is
equal to register \regt{} according to an unsigned comparison, and 0 otherwise.
\vspace{3ex}

\noindent
{\bf Set on equal to, unsigned, immediate}\\
\noindent
\texttt{sequi \regd, \regs, Immediate}
\itypeinsn{0011}{\regd}{\regs}{1001}{Immediate}
Set register \regd{} to 1 if register \regs{} is
equal to the zero-extended immediate according to an unsigned comparison, and 0 otherwise.
\vspace{3ex}

\noindent
{\bf Set on not equal to}\\
\noindent
\texttt{sne \regd, \regs, \regt}
\rtypeinsn{0010}{\regd}{\regs}{1010}{\regt}
Set register \regd{} to 1 if register \regs{} is
not equal to register \regt{} according to a signed comparison, and 0 otherwise.
\vspace{3ex}

\noindent
{\bf Set on not equal to immediate}\\
\noindent
\texttt{snei \regd, \regs, Immediate}
\itypeinsn{0011}{\regd}{\regs}{1010}{Immediate}
Set register \regd{} to 1 if register \regs{} is
not equal to the sign-extended immediate according to a signed comparison, and 0 otherwise.
\vspace{3ex}
\newpage

\noindent
{\bf Set on not equal to, unsigned}\\
\noindent
\texttt{sneu \regd, \regs, \regt}
\rtypeinsn{0010}{\regd}{\regs}{1011}{\regt}
Set register \regd{} to 1 if register \regs{} is
not equal to register \regt{} according to an unsigned comparison, and 0 otherwise.
\vspace{3ex}

\noindent
{\bf Set on not equal to, unsigned, immediate}\\
\noindent
\texttt{sneui \regd, \regs, Immediate}
\itypeinsn{0011}{\regd}{\regs}{1011}{Immediate}
Set register \regd{} to 1 if register \regs{} is
not equal to the zero-extended immediate according to an unsigned comparison, and 0 otherwise.
\vspace{3ex}

\subsection{Branch instructions}

\noindent
{\bf Jump}\\
\noindent
\texttt{j Address}
\jtypeinsn{0100}{0000}{0000}{Address}
Unconditionally jump to the instruction whose address is given by the address field.
\vspace{3ex}

\noindent
{\bf Jump to register}\\
\noindent
\texttt{jr \regs}
\jtypeinsn{0101}{0000}{\regs}{0000 0000 0000 0000 0000}
Unconditionally jump to the instruction whose
address is in the least significant 20 bits of register \regs{}.
\vspace{3ex}

\noindent
{\bf Jump and link}\\
\noindent
\texttt{jal Address}
\jtypeinsn{0110}{0000}{0000}{Address}
Unconditionally jump to the instruction whose address is given by the address field.
Save the address of the next instruction in register \texttt{\$ra}.
\vspace{3ex}
\newpage

\noindent
{\bf Jump and link register}\\
\noindent
\texttt{jalr \regs}
\jtypeinsn{0111}{0000}{\regs}{0000 0000 0000 0000 0000}
Unconditionally jump to the instruction whose
address is in the least significant 20 bits of register \regs{}.
Save the address of the next instruction in register \texttt{\$ra}.
\vspace{3ex}

\noindent
{\bf Branch on equal to zero}\\
\noindent
\texttt{beqz \regs, Offset}
\jtypeinsn{1010}{0000}{\regs}{Offset}
Conditionally branch the number of instructions specified by the
sign-extended offset if register \regs{} is equal to 0.
\vspace{3ex}

\noindent
{\bf Branch on not equal to zero}\\
\noindent
\texttt{bnez \regs, Offset}
\jtypeinsn{1011}{0000}{\regs}{Offset}
Conditionally branch the number of instructions specified by the
sign-extended offset if register \regs{} is not equal to 0.
\vspace{3ex}

\subsection{Memory instructions}

\noindent
{\bf Load word}\\
\noindent
\texttt{lw \regd, Offset(\regs{})}
\jtypeinsn{1000}{\regd}{\regs}{Offset}
Add the contents of register \regs{} and the sign-extended offset to
give an effective address. Load the contents of the memory location
specified by this effective address into register \regd{}.
\vspace{3ex}

\noindent
{\bf Store word}\\
\noindent
\texttt{sw \regd, Offset(\regs{})}
\jtypeinsn{1001}{\regd}{\regs}{Offset}
Add the contents of register \regs{} and the sign-extended offset to
give an effective address. Store the contents of register \regd{} into
the memory location specified by this effective address.
\vspace{3ex}
\newpage

\subsection{Special instructions}

\noindent
{\bf Move general register to special register}\\
\noindent
\texttt{movgs \regd, \regs}
\itypeinsn{0011}{\regd}{\regs}{1100}{0000 0000 0000 0000}
Copy the contents of general purpose register \regs{} into special purpose register \regd{}.
\vspace{3ex}

\noindent
{\bf Move special register to general register}\\
\noindent
\texttt{movsg \regd, \regs}
\itypeinsn{0011}{\regd}{\regs}{1101}{0000 0000 0000 0000}
Copy the contents of special purpose register \regs{} into general purpose register \regd{}.
\vspace{3ex}

\noindent
{\bf Break}\\
\noindent
\texttt{break}
\itypeinsn{0010}{0000}{0000}{1100}{0000 0000 0000 0000}
Generate a breakpoint exception, transferring control to the exception handler.
\vspace{3ex}

\noindent
{\bf System call}\\
\noindent
\texttt{syscall}
\itypeinsn{0010}{0000}{0000}{1101}{0000 0000 0000 0000}
Generate a system call exception, transferring control to the exception handler.
\vspace{3ex}

\noindent
{\bf Return from exception}\\
\noindent
\texttt{rfe}
\itypeinsn{0010}{0000}{0000}{1110}{0000 0000 0000 0000}
Restore the saved interrupt enable and kernel/user mode bits and jump to
the instruction at the address specified in the special register \texttt{\$ear}.
\vspace{3ex}

\end{document}


\begin{comment}
\newcommand\jtypeinsn[4]{
	\vspace{-4ex}
	\begin{center}
	\begin{tabular}{|c|c|c|c|}
	\multicolumn{1}{@{}p{8ex}}{} & \multicolumn{1}{@{}p{8ex}}{}
	& \multicolumn{1}{@{}p{8ex}}{} & \multicolumn{1}{@{}p{40ex}}{}\\ 
	\hline
	\texttt{#1} & \texttt{#2} & \texttt{#3} & \texttt{#4}\\
	\hline
	\multicolumn{1}{c}{\small 4} & \multicolumn{1}{c}{\small 4}
	& \multicolumn{1}{c}{\small 4} & \multicolumn{1}{c}{\small 20}\\
	\end{tabular}
	\end{center}
	}

\newcommand\jtypeinsnformat{
	\vspace{-4ex}
	\begin{center}
	\begin{tabular}{|c|c|c|c|}
	\multicolumn{1}{@{}p{8ex}}{} & \multicolumn{1}{@{}p{8ex}}{}
	& \multicolumn{1}{@{}p{8ex}}{} & \multicolumn{1}{@{}p{40ex}}{}\\ 
	\multicolumn{1}{c}{4 bits} & \multicolumn{1}{c}{4 bits}
	& \multicolumn{1}{c}{4 bits} & \multicolumn{1}{c}{20 bits}\\
	\hline
	\texttt{OPcode} & \texttt{\regd} & \texttt{\regs} & \texttt{Address / Offset}\\
	\hline
	\end{tabular}
	\end{center}
	}

\newcommand\rtypeinsn[5]{
	\vspace{-4ex}
	\begin{center}
	\begin{tabular}{|c|c|c|c|c|c|}
	\multicolumn{1}{@{}p{8ex}}{} & \multicolumn{1}{@{}p{8ex}}{}
	& \multicolumn{1}{@{}p{8ex}}{} & \multicolumn{1}{@{}p{8ex}}{}
	& \multicolumn{1}{@{}p{24ex}}{} & \multicolumn{1}{@{}p{8ex}}{}\\ 
	\hline
	\texttt{#1} & \texttt{#2} & \texttt{#3} & \texttt{#4} & \texttt{0000 0000 0000} & \texttt{#5}\\
	\hline
	\multicolumn{1}{c}{\small 4} & \multicolumn{1}{c}{\small 4}
	& \multicolumn{1}{c}{\small 4} & \multicolumn{1}{c}{\small 4}
	& \multicolumn{1}{c}{\small 12} & \multicolumn{1}{c}{\small 4}\\
	\end{tabular}
	\end{center}
	}

\newcommand\rtypeinsnformat{
	\vspace{-4ex}
	\begin{center}
	\begin{tabular}{|c|c|c|c|c|c|}
	\multicolumn{1}{@{}p{8ex}}{} & \multicolumn{1}{@{}p{8ex}}{}
	& \multicolumn{1}{@{}p{8ex}}{} & \multicolumn{1}{@{}p{8ex}}{}
	& \multicolumn{1}{@{}p{24ex}}{} & \multicolumn{1}{@{}p{8ex}}{}\\ 
	\multicolumn{1}{c}{4 bits} & \multicolumn{1}{c}{4 bits}
	& \multicolumn{1}{c}{4 bits} & \multicolumn{1}{c}{4 bits}
	& \multicolumn{1}{c}{12 bits} & \multicolumn{1}{c}{4 bits}\\
	\hline
	\texttt{OPcode} & \texttt{\regd} & \texttt{\regs} & \texttt{Func} & \texttt{0000 0000 0000} & \texttt{\regt}\\
	\hline
	\end{tabular}
	\end{center}
	}

\newcommand\itypeinsn[5]{
	\vspace{-4ex}
	\begin{center}
	\begin{tabular}{|c|c|c|c|c|}
	\multicolumn{1}{@{}p{8ex}}{} & \multicolumn{1}{@{}p{8ex}}{}
	& \multicolumn{1}{@{}p{8ex}}{} & \multicolumn{1}{@{}p{8ex}}{}
	& \multicolumn{1}{@{}p{32ex}}{}\\ 
	\hline
	\texttt{#1} & \texttt{#2} & \texttt{#3} & \texttt{#4} & \texttt{#5}\\
	\hline
	\multicolumn{1}{c}{\small 4} & \multicolumn{1}{c}{\small 4}
	& \multicolumn{1}{c}{\small 4} & \multicolumn{1}{c}{\small 4}
	& \multicolumn{1}{c}{\small 16}\\
	\end{tabular}
	\end{center}
	}

\newcommand\itypeinsnformat{
	\vspace{-4ex}
	\begin{center}
	\begin{tabular}{|c|c|c|c|c|}
	\multicolumn{1}{@{}p{8ex}}{} & \multicolumn{1}{@{}p{8ex}}{}
	& \multicolumn{1}{@{}p{8ex}}{} & \multicolumn{1}{@{}p{8ex}}{}
	& \multicolumn{1}{@{}p{32ex}}{}\\ 

	\multicolumn{1}{c}{4 bits} & \multicolumn{1}{c}{4 bits}
	& \multicolumn{1}{c}{4 bits} & \multicolumn{1}{c}{4 bits}
	& \multicolumn{1}{c}{16 bits}\\

	\hline
	\texttt{OPcode} & \texttt{\regd} & \texttt{\regs} & \texttt{Func} & \texttt{Immediate}\\
	\hline
	\end{tabular}
	\end{center}
	}


\begin{document}


\setcounter{secnumdepth}{0}
%\pagenumbering{roman}

\section{WRAMP General Purpose Registers}

The WRAMP general purpose register file consists of 16 registers, each
being 32 bits wide. The hardware imposes special uses on only two of these.
Certain software register use conventions have been applied to some of the
remaining registers, but in essence they remain true general purpose
registers, as the hardware does not restrict their use.

\vspace{-4ex}
\begin{figure}[h]
\begin{center}
\begin{tabular}{|l|l|}
\multicolumn{1}{@{}p{4ex}}{}
& \multicolumn{1}{@{}p{50ex}}{}\\ 
\hline
Register &Description\\
\hline
\texttt{\$0} & Hardwired zero\\
\texttt{\$1} - \texttt{\$13} & General purpose registers\\
\texttt{\$sp} & Stack pointer\\
\texttt{\$ra} & Return address register\\
\hline
\end{tabular}
\end{center}
\caption{The WRAMP General Purpose Registers}
\label{wramp_regs}
\end{figure}

Register zero (denoted \texttt{\$0}) always contains the value
zero. Any writes to this register have the value discarded. This provides a
constant source of zero that can be used for comparing and initialising
registers.

The fourteenth register is denoted \texttt{\$sp}. This register is defined
by the conventions to be the stack pointer. While the hardware imposes no
special conditions on this register, failure to follow this convention may
affect the ability of code to interoperate with other software.

The fifteenth register is denoted \texttt{\$ra}. It is defined to be the
subroutine return address register.  When a jump and link instruction is
executed this register is loaded with the address of the next instruction
after the jump and link. A return from subroutine is performed by executing
a jump to register \texttt{\$ra}, ie. \texttt{jr \$ra}.

\section{WRAMP Instruction Set Architecture}

This section contains the details of the WRAMP instruction set.
All machine instructions are listed, with their encoding and a brief description of
their function.
The instructions are grouped into arithmetic instructions, bitwise instructions,
test instructions, branch instructions, memory instructions, and special instructions.

Each CPU instruction is a word (32 bits) in length. An instruction is encoded in
one of the three formats shown in figure \ref{insn_encoding_formats}.

\newpage
% This is the table with the three instruction encoding formats
\begin{figure}[h]
{\bf I-Type instruction}
\itypeinsnformat
{\bf R-Type instruction}
\rtypeinsnformat
{\bf J-Type instruction}
\jtypeinsnformat
\vspace{1ex}
\begin{tabbing}
xxxx\=field name xxxxxxxx\=description \kill
\>\texttt{OPCode}\>4 bit operation code\\
\>\texttt{\regd{}}\>4 bit destination register specifier\\
\>\texttt{\regs{}}\>4 bit source register specifier\\
\>\texttt{\regt{}}\>4 bit source register specifier\\
\>\texttt{Func}\>4 bit function specifier\\
\>\texttt{Immediate}\>16 bit immediate field\\
\>\texttt{Address / Offset}\>20 bit absolute or relative address field
\end{tabbing}
\caption{WRAMP Instruction encoding formats}
\label{insn_encoding_formats}
\end{figure}


\subsection{Arithmetic Instructions}
\noindent
{\bf Addition}\\
\noindent
\texttt{add \regd, \regs, \regt}
\rtypeinsn{0000}{\regd}{\regs}{0000}{\regt}
Put the sum of register \regs{} and register \regt{}
into register \regd{}. Generate an overflow exception on signed overflow.
\vspace{3ex}

\noindent
{\bf Addition, immediate}\\
\noindent
\texttt{addi \regd, \regs, Immediate}
\itypeinsn{0001}{\regd}{\regs}{0000}{Immediate}
Put the sum of register \regs{} and the sign-extended immediate
into register \regd{}. Generate an overflow exception on signed overflow.
\vspace{3ex}
\newpage

\noindent
{\bf Addition, unsigned}\\
\noindent
\texttt{addu \regd, \regs, \regt}
\rtypeinsn{0000}{\regd}{\regs}{0001}{\regt}
Put the sum of register \regs{} and register \regt{}
into register \regd{}. Generate an overflow exception on unsigned overflow.
\vspace{3ex}

\noindent
{\bf Addition, unsigned, immediate}\\
\noindent
\texttt{addui \regd, \regs, Immediate}
\itypeinsn{0001}{\regd}{\regs}{0001}{Immediate}
Put the sum of register \regs{} and the zero-extended immediate
into register \regd{}. Generate an overflow exception on unsigned overflow.
\vspace{3ex}

\noindent
{\bf Subtraction}\\
\noindent
\texttt{sub \regd, \regs, \regt}
\rtypeinsn{0000}{\regd}{\regs}{0010}{\regt}
Put the difference of register \regs{} and register \regt{}
into register \regd{}. Generate an overflow exception on signed overflow.
\vspace{3ex}

\noindent
{\bf Subtraction, immediate}\\
\noindent
\texttt{subi \regd, \regs, Immediate}
\itypeinsn{0001}{\regd}{\regs}{0010}{Immediate}
Put the difference of register \regs{} and the sign-extended immediate
into register \regd{}. Generate an overflow exception on signed overflow.
\vspace{3ex}

\noindent
{\bf Subtraction, unsigned}\\
\noindent
\texttt{subu \regd, \regs, \regt}
\rtypeinsn{0000}{\regd}{\regs}{0011}{\regt}
Put the difference of register \regs{} and register \regt{}
into register \regd{}. Generate an overflow exception on unsigned overflow.
\vspace{3ex}
\newpage

\noindent
{\bf Subtraction, unsigned, immediate}\\
\noindent
\texttt{subui \regd, \regs, Immediate}
\itypeinsn{0001}{\regd}{\regs}{0011}{Immediate}
Put the difference of register \regs{} and the zero-extended immediate
into register \regd{}. Generate an overflow exception on unsigned overflow.
\vspace{3ex}

\noindent
{\bf Multiplication}\\
\noindent
\texttt{mult \regd, \regs, \regt}
\rtypeinsn{0000}{\regd}{\regs}{0100}{\regt}
Put the product of the signed multiplication of register \regs{} and register \regt{}
into register \regd{}. Generate an overflow exception on signed overflow.
\vspace{3ex}

\noindent
{\bf Multiplication, immediate}\\
\noindent
\texttt{multi \regd, \regs, Immediate}
\itypeinsn{0001}{\regd}{\regs}{0100}{Immediate}
Put the product of the signed multiplication of register \regs{} and the sign-extended immediate
into register \regd{}. Generate an overflow exception on signed overflow.
\vspace{3ex}

\noindent
{\bf Multiplication, unsigned}\\
\noindent
\texttt{multu \regd, \regs, \regt}
\rtypeinsn{0000}{\regd}{\regs}{0101}{\regt}
Put the product of the unsigned multiplication of register \regs{} and register \regt{}
into register \regd{}. Generate an overflow exception on unsigned overflow.
\vspace{3ex}

\noindent
{\bf Multiplication, unsigned, immediate}\\
\noindent
\texttt{multui \regd, \regs, Immediate}
\itypeinsn{0001}{\regd}{\regs}{0101}{Immediate}
Put the product of the unsigned multiplication of register \regs{} and the zero-extended immediate
into register \regd{}. Generate an overflow exception on unsigned overflow.
\vspace{3ex}
\newpage

\noindent
{\bf Division}\\
\noindent
\texttt{div \regd, \regs, \regt}
\rtypeinsn{0000}{\regd}{\regs}{0110}{\regt}
Put the result of the signed integer division of register \regs{} by register \regt{}
into register \regd{}. Generate a divide-by-zero exception if the contents of \regt{} is zero.
\vspace{3ex}

\noindent
{\bf Division, immediate}\\
\noindent
\texttt{divi \regd, \regs, Immediate}
\itypeinsn{0001}{\regd}{\regs}{0110}{Immediate}
Put the result of the signed integer division of register \regs{} by the sign-extended immediate
into register \regd{}. Generate a divide-by-zero exception if the immediate value is zero.
\vspace{3ex}

\noindent
{\bf Division, unsigned}\\
\noindent
\texttt{divu \regd, \regs, \regt}
\rtypeinsn{0000}{\regd}{\regs}{0111}{\regt}
Put the result of the unsigned division of register \regs{} by register \regt{}
into register \regd{}. Generate a divide-by-zero exception if the contents of \regt{} is zero.
\vspace{3ex}

\noindent
{\bf Division, unsigned, immediate}\\
\noindent
\texttt{divui \regd, \regs, Immediate}
\itypeinsn{0001}{\regd}{\regs}{0111}{Immediate}
Put the result of the unsigned division of register \regs{} by the zero-extended immediate
into register \regd{}. Generate a divide-by-zero exception if the immediate value is zero. 
\vspace{3ex}

\noindent
{\bf Remainder}\\
\noindent
\texttt{rem \regd, \regs, \regt}
\rtypeinsn{0000}{\regd}{\regs}{1000}{\regt}
Put the remainder of the signed division of register \regs{} by register \regt{}
into register \regd{}. Generate a divide-by-zero exception if the contents of \regt{} is zero.
\vspace{3ex}
\newpage

\noindent
{\bf Remainder, immediate}\\
\noindent
\texttt{remi \regd, \regs, Immediate}
\itypeinsn{0001}{\regd}{\regs}{1000}{Immediate}
Put the remainder of the signed division of register \regs{} by the sign-extended immediate
into register \regd{}. Generate a divide-by-zero exception if the immediate value is zero.
\vspace{3ex}

\noindent
{\bf Remainder, unsigned}\\
\noindent
\texttt{remu \regd, \regs, \regt}
\rtypeinsn{0000}{\regd}{\regs}{1001}{\regt}
Put the remainder of the unsigned division of register \regs{} by the register \regt{}
into register \regd{}. Generate a divide-by-zero exception if the contents of \regt{} is zero.
\vspace{3ex}

\noindent
{\bf Remainder, unsigned, immediate}\\
\noindent
\texttt{remui \regd, \regs, Immediate}
\itypeinsn{0001}{\regd}{\regs}{1001}{Immediate}
Put the remainder of the unsigned division of register \regs{} by the zero-extended immediate
into register \regd{}. Generate a divide-by-zero exception if the immediate value is zero.
\vspace{3ex}

\noindent
{\bf Load high immediate}\\
\noindent
\texttt{lhi \regd, Immediate}
\itypeinsn{0011}{\regd}{0000}{1110}{Immediate}
Put the 16 bit immediate into the upper 16 bits of register \regd{},
and set the lower 16 bits to zero.
\vspace{3ex}

\noindent
{\bf Load address}\\
\noindent
\texttt{la \regd, Address}
\jtypeinsn{1100}{\regd}{0000}{Address}
Put the zero-extended 20 bit address into register \regd{}.
\vspace{3ex}
\newpage

\subsection{Bitwise instructions}

\noindent
{\bf And}\\
\noindent
\texttt{and \regd, \regs, \regt}
\rtypeinsn{0000}{\regd}{\regs}{1011}{\regt}
Put the result of the logical AND of registers \regs{} and \regt{} into register \regd{}.
\vspace{3ex}

\noindent
{\bf And, immediate}\\
\noindent
\texttt{andi \regd, \regs, Immediate}
\itypeinsn{0001}{\regd}{\regs}{1011}{Immediate}
Put the result of the logical AND of register \regs{} and the zero-extended immediate into register \regd{}.
\vspace{3ex}

\noindent
{\bf Or}\\
\noindent
\texttt{or \regd, \regs, \regt}
\rtypeinsn{0000}{\regd}{\regs}{1101}{\regt}
Put the result of the logical OR of registers \regs{} and \regt{} into register \regd{}.
\vspace{3ex}

\noindent
{\bf Or, immediate}\\
\noindent
\texttt{ori \regd, \regs, Immediate}
\itypeinsn{0001}{\regd}{\regs}{1101}{Immediate}
Put the result of the logical OR of register \regs{} and the zero-extended immediate into register \regd{}.
\vspace{3ex}

\noindent
{\bf Xor}\\
\noindent
\texttt{xor \regd, \regs, \regt}
\rtypeinsn{0000}{\regd}{\regs}{1111}{\regt}
Put the result of the logical exclusive-OR of registers \regs{} and \regt{} into register \regd{}.
\vspace{3ex}
\newpage

\noindent
{\bf Xor, immediate}\\
\noindent
\texttt{xori \regd, \regs, Immediate}
\itypeinsn{0001}{\regd}{\regs}{1111}{Immediate}
Put the result of the logical exclusive-OR of register \regs{} and the zero-extended immediate into register \regd{}.
\vspace{3ex}

\noindent
{\bf Shift left logical}\\
\noindent
\texttt{sll \regd, \regs, \regt}
\rtypeinsn{0000}{\regd}{\regs}{1010}{\regt}
Shift the value in register \regs{} left by the unsigned value given by the
least significant 5 bits of register \regt{}, and put the result in register \regd{},
inserting zeros into the low order bits.
\vspace{3ex}

\noindent
{\bf Shift left logical, immediate}\\
\noindent
\texttt{slli \regd, \regs, Immediate}
\itypeinsn{0001}{\regd}{\regs}{1010}{Immediate}
Shift the value in register \regs{} left by the unsigned value given by the
least significant 5 bits of the immediate, and put the result in register \regd{},
inserting zeros into the low order bits.
\vspace{3ex}

\noindent
{\bf Shift right logical}\\
\noindent
\texttt{srl \regd, \regs, \regt}
\rtypeinsn{0000}{\regd}{\regs}{1100}{\regt}
Shift the value in register \regs{} right by the unsigned value given by the
least significant 5 bits of register \regt{}, and place the result in register \regd{},
inserting zeros into the high order bits.
\vspace{3ex}

\noindent
{\bf Shift right logical, immediate}\\
\noindent
\texttt{srli \regd, \regs, Immediate}
\itypeinsn{0001}{\regd}{\regs}{1100}{Immediate}
Shift the value in register \regs{} right by the unsigned value given by the
least significant 5 bits of the immediate, and place the result in register \regd{},
inserting zeros into the high order bits.
\vspace{3ex}
\newpage

\noindent
{\bf Shift right arithmetic}\\
\noindent
\texttt{sra \regd, \regs, \regt}
\rtypeinsn{0000}{\regd}{\regs}{1110}{\regt}
Shift the value in register \regs{} right by the unsigned value given by the
least significant 5 bits of register \regt{}, and place the result in register \regd{},
sign-extending the high order bits.
\vspace{3ex}

\noindent
{\bf Shift right arithmetic, immediate}\\
\noindent
\texttt{srai \regd, \regs, Immediate}
\itypeinsn{0001}{\regd}{\regs}{1110}{Immediate}
Shift the value in register \regs{} right by the unsigned value given by the
least significant 5 bits of the immediate, and place the result in register \regd{},
sign-extending the high order bits.
\vspace{3ex}

\subsection{Test instructions}

\noindent
{\bf Set on less than}\\
\noindent
\texttt{slt \regd, \regs, \regt}
\rtypeinsn{0010}{\regd}{\regs}{0000}{\regt}
Set register \regd{} to 1 if register \regs{} is
less than register \regt{} according to a signed comparison, and 0 otherwise.
\vspace{3ex}

\noindent
{\bf Set on less than immediate}\\
\noindent
\texttt{slti \regd, \regs, Immediate}
\itypeinsn{0011}{\regd}{\regs}{0000}{Immediate}
Set register \regd{} to 1 if register \regs{} is
less than the sign-extended immediate according to a signed comparison, and 0 otherwise.
\vspace{3ex}
\newpage

\noindent
{\bf Set on less than, unsigned}\\
\noindent
\texttt{sltu \regd, \regs, \regt}
\rtypeinsn{0010}{\regd}{\regs}{0001}{\regt}
Set register \regd{} to 1 if register \regs{} is
less than register \regt{} according to an unsigned comparison, and 0 otherwise.
\vspace{3ex}

\noindent
{\bf Set on less than, unsigned, immediate}\\
\noindent
\texttt{sltui \regd, \regs, Immediate}
\itypeinsn{0011}{\regd}{\regs}{0001}{Immediate}
Set register \regd{} to 1 if register \regs{} is
less than the zero-extended immediate according to an unsigned comparison, and 0 otherwise.
\vspace{3ex}

\noindent
{\bf Set on greater than}\\
\noindent
\texttt{sgt \regd, \regs, \regt}
\rtypeinsn{0010}{\regd}{\regs}{0010}{\regt}
Set register \regd{} to 1 if register \regs{} is
greater than register \regt{} according to a signed comparison, and 0 otherwise.
\vspace{3ex}

\noindent
{\bf Set on greater than, immediate}\\
\noindent
\texttt{sgti \regd, \regs, Immediate}
\itypeinsn{0011}{\regd}{\regs}{0010}{Immediate}
Set register \regd{} to 1 if register \regs{} is
greater than the sign-extended immediate according to a signed comparison, and 0 otherwise.
\vspace{3ex}

\noindent
{\bf Set on greater than, unsigned}\\
\noindent
\texttt{sgtu \regd, \regs, \regt}
\rtypeinsn{0010}{\regd}{\regs}{0011}{\regt}
Set register \regd{} to 1 if register \regs{} is
greater than register \regt{} according to an unsigned comparison, and 0 otherwise.
\vspace{3ex}
\newpage

\noindent
{\bf Set on greater than, unsigned, immediate}\\
\noindent
\texttt{sgtui \regd, \regs, Immediate}
\itypeinsn{0011}{\regd}{\regs}{0011}{Immediate}
Set register \regd{} to 1 if register \regs{} is
greater than the zero-extended immediate according to an unsigned comparison, and 0 otherwise.
\vspace{3ex}

\noindent
{\bf Set on less than or equal to}\\
\noindent
\texttt{sle \regd, \regs, \regt}
\rtypeinsn{0010}{\regd}{\regs}{0100}{\regt}
Set register \regd{} to 1 if register \regs{} is
less than or equal to register \regt{} according to a signed comparison, and 0 otherwise.
\vspace{3ex}

\noindent
{\bf Set on less than or equal to, immediate}\\
\noindent
\texttt{slei \regd, \regs, Immediate}
\itypeinsn{0011}{\regd}{\regs}{0100}{Immediate}
Set register \regd{} to 1 if register \regs{} is
less than or equal to the sign-extended immediate according to a signed comparison, and 0 otherwise.
\vspace{3ex}

\noindent
{\bf Set on less than or equal to, unsigned}\\
\noindent
\texttt{sleu \regd, \regs, \regt}
\rtypeinsn{0010}{\regd}{\regs}{0101}{\regt}
Set register \regd{} to 1 if register \regs{} is
less than or equal to register \regt{} according to an unsigned comparison, and 0 otherwise.
\vspace{3ex}

\noindent
{\bf Set on less than or equal to, unsigned, immediate}\\
\noindent
\texttt{sleui \regd, \regs, Immediate}
\itypeinsn{0011}{\regd}{\regs}{0101}{Immediate}
Set register \regd{} to 1 if register \regs{} is
less than or equal to the zero-extended immediate according to an unsigned comparison, and 0 otherwise.
\vspace{3ex}
\newpage

\noindent
{\bf Set on greater than or equal to}\\
\noindent
\texttt{sge \regd, \regs, \regt}
\rtypeinsn{0010}{\regd}{\regs}{0110}{\regt}
Set register \regd{} to 1 if register \regs{} is
greater than or equal to register \regt{} according to a signed comparison, and 0 otherwise.
\vspace{3ex}

\noindent
{\bf Set on greater than or equal to immediate}\\
\noindent
\texttt{sgei \regd, \regs, Immediate}
\itypeinsn{0011}{\regd}{\regs}{0110}{Immediate}
Set register \regd{} to 1 if register \regs{} is
greater than or equal to the sign-extended immediate according to a signed comparison, and 0 otherwise.
\vspace{3ex}

\noindent
{\bf Set on greater than or equal to, unsigned}\\
\noindent
\texttt{sgeu \regd, \regs, \regt}
\rtypeinsn{0010}{\regd}{\regs}{0111}{\regt}
Set register \regd{} to 1 if register \regs{} is
greater than or equal to register \regt{} according to an unsigned comparison, and 0 otherwise.
\vspace{3ex}

\noindent
{\bf Set on greater than or equal to, unsigned, immediate}\\
\noindent
\texttt{sgeui \regd, \regs, Immediate}
\itypeinsn{0011}{\regd}{\regs}{0111}{Immediate}
Set register \regd{} to 1 if register \regs{} is
greater than or equal to the zero-extended immediate according to an unsigned comparison, and 0 otherwise.
\vspace{3ex}

\noindent
{\bf Set on equal to}\\
\noindent
\texttt{seq \regd, \regs, \regt}
\rtypeinsn{0010}{\regd}{\regs}{1000}{\regt}
Set register \regd{} to 1 if register \regs{} is
equal to register \regt{} according to a signed comparison, and 0 otherwise.
\vspace{3ex}
\newpage

\noindent
{\bf Set on equal to immediate}\\
\noindent
\texttt{seqi \regd, \regs, Immediate}
\itypeinsn{0011}{\regd}{\regs}{1000}{Immediate}
Set register \regd{} to 1 if register \regs{} is
equal to the sign-extended immediate according to a signed comparison, and 0 otherwise.
\vspace{3ex}

\noindent
{\bf Set on equal to, unsigned}\\
\noindent
\texttt{sequ \regd, \regs, \regt}
\rtypeinsn{0010}{\regd}{\regs}{1001}{\regt}
Set register \regd{} to 1 if register \regs{} is
equal to register \regt{} according to an unsigned comparison, and 0 otherwise.
\vspace{3ex}

\noindent
{\bf Set on equal to, unsigned, immediate}\\
\noindent
\texttt{sequi \regd, \regs, Immediate}
\itypeinsn{0011}{\regd}{\regs}{1001}{Immediate}
Set register \regd{} to 1 if register \regs{} is
equal to the zero-extended immediate according to an unsigned comparison, and 0 otherwise.
\vspace{3ex}

\noindent
{\bf Set on not equal to}\\
\noindent
\texttt{sne \regd, \regs, \regt}
\rtypeinsn{0010}{\regd}{\regs}{1010}{\regt}
Set register \regd{} to 1 if register \regs{} is
not equal to register \regt{} according to a signed comparison, and 0 otherwise.
\vspace{3ex}

\noindent
{\bf Set on not equal to immediate}\\
\noindent
\texttt{snei \regd, \regs, Immediate}
\itypeinsn{0011}{\regd}{\regs}{1010}{Immediate}
Set register \regd{} to 1 if register \regs{} is
not equal to the sign-extended immediate according to a signed comparison, and 0 otherwise.
\vspace{3ex}
\newpage

\noindent
{\bf Set on not equal to, unsigned}\\
\noindent
\texttt{sneu \regd, \regs, \regt}
\rtypeinsn{0010}{\regd}{\regs}{1011}{\regt}
Set register \regd{} to 1 if register \regs{} is
not equal to register \regt{} according to an unsigned comparison, and 0 otherwise.
\vspace{3ex}

\noindent
{\bf Set on not equal to, unsigned, immediate}\\
\noindent
\texttt{sneui \regd, \regs, Immediate}
\itypeinsn{0011}{\regd}{\regs}{1011}{Immediate}
Set register \regd{} to 1 if register \regs{} is
not equal to the zero-extended immediate according to an unsigned comparison, and 0 otherwise.
\vspace{3ex}

\subsection{Branch instructions}

\noindent
{\bf Jump}\\
\noindent
\texttt{j Address}
\jtypeinsn{0100}{0000}{0000}{Address}
Unconditionally jump to the instruction whose address is given by the address field.
\vspace{3ex}

\noindent
{\bf Jump to register}\\
\noindent
\texttt{jr \regs}
\jtypeinsn{0101}{0000}{\regs}{0000 0000 0000 0000 0000}
Unconditionally jump to the instruction whose
address is in the least significant 20 bits of register \regs{}.
\vspace{3ex}

\noindent
{\bf Jump and link}\\
\noindent
\texttt{jal Address}
\jtypeinsn{0110}{0000}{0000}{Address}
Unconditionally jump to the instruction whose address is given by the address field.
Save the address of the next instruction in register \texttt{\$ra}.
\vspace{3ex}
\newpage

\noindent
{\bf Jump and link register}\\
\noindent
\texttt{jalr \regs}
\jtypeinsn{0111}{0000}{\regs}{0000 0000 0000 0000 0000}
Unconditionally jump to the instruction whose
address is in the least significant 20 bits of register \regs{}.
Save the address of the next instruction in register \texttt{\$ra}.
\vspace{3ex}

\noindent
{\bf Branch on equal to zero}\\
\noindent
\texttt{beqz \regs, Offset}
\jtypeinsn{1010}{0000}{\regs}{Offset}
Conditionally branch the number of instructions specified by the
sign-extended offset if register \regs{} is equal to 0.
\vspace{3ex}

\noindent
{\bf Branch on not equal to zero}\\
\noindent
\texttt{bnez \regs, Offset}
\jtypeinsn{1011}{0000}{\regs}{Offset}
Conditionally branch the number of instructions specified by the
sign-extended offset if register \regs{} is not equal to 0.
\vspace{3ex}

\subsection{Memory instructions}

\noindent
{\bf Load word}\\
\noindent
\texttt{lw \regd, Offset(\regs{})}
\jtypeinsn{1000}{\regd}{\regs}{Offset}
Add the contents of register \regs{} and the sign-extended offset to
give an effective address. Load the contents of the memory location
specified by this effective address into register \regd{}.
\vspace{3ex}

\noindent
{\bf Store word}\\
\noindent
\texttt{sw \regd, Offset(\regs{})}
\jtypeinsn{1001}{\regd}{\regs}{Offset}
Add the contents of register \regs{} and the sign-extended offset to
give an effective address. Store the contents of register \regd{} into
the memory location specified by this effective address.
\vspace{3ex}
\newpage

\subsection{Special instructions}

\noindent
{\bf Move general register to special register}\\
\noindent
\texttt{movgs \regd, \regs}
\itypeinsn{0011}{\regd}{\regs}{1100}{0000 0000 0000 0000}
Copy the contents of general purpose register \regs{} into special purpose register \regd{}.
\vspace{3ex}

\noindent
{\bf Move special register to general register}\\
\noindent
\texttt{movsg \regd, \regs}
\itypeinsn{0011}{\regd}{\regs}{1101}{0000 0000 0000 0000}
Copy the contents of special purpose register \regs{} into general purpose register \regd{}.
\vspace{3ex}

\noindent
{\bf Break}\\
\noindent
\texttt{break}
\itypeinsn{0010}{0000}{0000}{1100}{0000 0000 0000 0000}
Generate a breakpoint exception, transferring control to the exception handler.
\vspace{3ex}

\noindent
{\bf System call}\\
\noindent
\texttt{syscall}
\itypeinsn{0010}{0000}{0000}{1101}{0000 0000 0000 0000}
Generate a system call exception, transferring control to the exception handler.
\vspace{3ex}

\noindent
{\bf Return from exception}\\
\noindent
\texttt{rfe}
\itypeinsn{0010}{0000}{0000}{1110}{0000 0000 0000 0000}
Restore the saved interrupt enable and kernel/user mode bits and jump to
the instruction at the address specified in the special register \texttt{\$ear}.
\vspace{3ex}
\end{comment}
%\end{document}
