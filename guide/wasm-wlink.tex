\setcounter{secnumdepth}{0}

\section{\wasm\ and \wlink}

\wasm\ and \wlink\ are the \BI{assembler} and \BI{linker} for WRAMP code,
respectively. \wasm\ receives a file containing WRAMP assembly code and
outputs an object file, while \wlink\ receives a set of object files
and outputs a linked program, in the Motorola S-record file format which
\WRAMPmon\ can receive.

An additional program, named \trim, can be used to convert an S-record
into a \filename{.mem} file, which can be used in the Xilinx Vivado Design
Suite when developing \WRAMPmon\ itself. \trim\ does not need to be
used for general WRAMP development, but is documented here for completeness.

This section details the usage and capabilities of these programs, as well as
some information on the file formats themselves.

\subsection{Command Line Usage}

The basic usage of each tool is quite simple. With the programs accessible by
a command line, their help notes can be seen by simply running them with no
arguments, such as in the following examples.

\begin{verbatim}
     $ wasm
     USAGE: wasm [-o output] file
     $ wlink
     USAGE: wlink [-Ttext address] [-Tdata address] [-[T|E]bss address] [-v] 
            [-o output] file1 file2 ...
\end{verbatim}

It can be seen that \wasm\ takes a single filename for a file containing
WRAMP assembly as a command-line argument, and will optionally accept a
\src{-o output} parameter, which allows you to specify an output file. By 
efault, the output file will be created in the same directory as the input
file and given the same name, but with the file extension replaced with
\filename{.o}. The convention is for WRAMP assembly files to be given the
\filename{.s} or \filename{.S} extension.

\wlink\ can accept more complicated arguments, but the basic usage is
the same as for \wasm. The optional \src{-o output} parameter will set
the output file in the same fashion as \wasm, but the file extension will
become \filename{.srec}. The rest of the arguments are considered input files,
which should all be object files as created by \wasm.

If \wlink\ is given a \src{-Ttext}, \src{-Tdata}, or \src{-[T|E]bss} argument, 
it will place the \BI{top} of the given segment at the address specified. For
example, if \src{-Ttext 0x10} is given, the first instruction in the
\src{.text} segment will be at address 0x10. The same applies for the
\src{.data} segment, and the \src{.bss} segment if \src{-Tbss} is given. If
\src{-Ebss} is given instead, the \src{.bss} segment will \BI{end} at the
address specified.

Finally, \wlink\ supports a \src{-v}, or \BI{verbose} option. Including this
option makes \wlink\ output additional information about the resulting
\filename{.srec} file. An example of a simple program that sets the values
of the LEDs and seven-segment displays to the state of the switches is shown
in Figure \ref{wlink-output}.

\wlink\ first prints a disassembly of the \src{.text} segment of the entire
program, after reordering the \src{.text} segments from each input file in the
order they were given on the command line. Each file has a separate header
specifying what address that segment begins at in the final program. The
disassembly is given, making it more easy to see which instruction ended up
where. This is useful when combined with \WRAMPmon's debugging commands,
detailed in Section \ref{intro:debugging}.

A display of the \src{.data} segments follows. This is a simple list of any
information placed in that section of each file, in the form it will appear
in memory. The example above shows the string \src{Hello!}. Finally, the
size of each file's \src{.bss} segment is displayed, followed by a summary
of each segment.

\begin{figure}[btp]
\begin{center}
\begin{tabular}{|p{15cm}|}
\hline
\begin{verbatim}
     $ wlink -v -o four-sevensegs.srec four-sevensegs.o
     file `four-sevensegs.o', starting : 0x00000, .text
     0x00000 : 81073000    lw	$1,0x73000($0)
     0x00001 : 9107300a    sw	$1,0x7300a($0)
     0x00002 : 121bf000    andi	$2,$1,0xf000
     0x00003 : 122c000c    srli	$2,$2,0x000c
     0x00004 : 131b0f00    andi	$3,$1,0x0f00
     0x00005 : 133c0008    srli	$3,$3,0x0008
     0x00006 : 141b00f0    andi	$4,$1,0x00f0
     0x00007 : 144c0004    srli	$4,$4,0x0004
     0x00008 : 151b000f    andi	$5,$1,0x000f
     0x00009 : 1a000001    addi	$10,$0,0x0001
     0x0000a : 9a073004    sw	$10,0x73004($0)
     0x0000b : 92073006    sw	$2,0x73006($0)
     0x0000c : 93073007    sw	$3,0x73007($0)
     0x0000d : 94073008    sw	$4,0x73008($0)
     0x0000e : 95073009    sw	$5,0x73009($0)
     0x0000f : 40000000    j	0x00000
     
     file `four-sevensegs.o', starting : 0x00010, .data
     0x00010 : 00000048    
     0x00011 : 00000065    
     0x00012 : 0000006c    
     0x00013 : 0000006c    
     0x00014 : 0000006f    
     0x00015 : 00000021    
     0x00016 : 00000000    

     file `four-sevensegs.o', starting : 0x00010, .bss : 0 words.

     entry point : 0x00000
     .text segment size = 0x00000010
     .data segment size = 0x00000007
     .bss segment size = 0x00000000
\end{verbatim}
\\
\hline
\end{tabular}
\end{center}
\caption{Example \wlink\ output for a simple program}
\label{wlink-output}
\end{figure}

\trim\ takes the same arguments as \wasm, but the input should be an
\src{.srec} file such as one generated by \wlink, and the output will default
to the \src{.mem} file extension.

\section{WRAMP File Formats}

There are three file formats which are used for all development of WRAMP
programs: \filename{.s}, \filename{.o}, and \filename{.srec}.
A \filename{.s} file contains WRAMP assembly source code, and is assembled into
a \filename{.o} \BI{object file}. One or several object files can be linked by
\wlink\ into an \filename{.srec} file, ready to be loaded into \WRAMPmon.

For projects containing mixed C and assembly code, it may be useful to follow
a slightly different convention. One such convention (used in the \WRAMPmon\ 
source code) is that assembly and object files which are not generated by a
tool, but are instead written or hand-picked by the user have a capitalised
file extension (\filename{.S} or \filename{.O}). This makes it clear which
files can be safely deleted, such as in a Makefile's \src{make clean} target.

\subsection{\wobj\ and \wdis}

Two tools exist which are able to inspect object and S-Record files: \wobj\
and \wdis. They should each be provided a single file of the correct format as
a command line argument.

\wobj\ will provide some details about object files, presented in a readable
format. The first section shows the \BI{symbol table}, which is a listing of
all global or undefined labels found in the source file. This could, for
example, be used to discover what functions a pre-assembled library offers.

The second section is the \BI{relocation table}, which shows where certain
parts of instructions will be replaced with other properties, such as the
address of a label. This will not show any unresolved labels.

Finally, a disassembly of the \src{.text} segment is displayed, in a similar
fashion to the disassemblies offered by \wlink\ and \WRAMPmon.

\wdis\ is able to disassemble an \filename{.srec} file. Its output is also
comparable to that of \wlink\ and \WRAMPmon.

